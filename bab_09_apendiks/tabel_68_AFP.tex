\phantomsection
\addcontentsline{lot}{section}{\protect\numberline{}Tabel 68 - Jumlah kasus AFP (Non Polio)}
\ra{1.3}
%Table generated by Excel2LaTeX from sheet '68'

{\centering
\begin{tabular}{rY{9em}Y{8em}Z{10em}Z{10em}}
    \multicolumn{5}{l}{Tabel 68}\\
    \multicolumn{5}{c}{JUMLAH KASUS AFP\textsuperscript{1} (NON POLIO) MENURUT KECAMATAN DAN PUSKESMAS}\\
    \multicolumn{5}{c}{KABUPATEN BELITUNG TIMUR}\\
    \multicolumn{5}{c}{TAHUN \tP}\\
    \toprule
    NO & KECAMATAN & PUSKESMAS & JUMLAH PENDUDUK < 15 TAHUN & JUMLAH KASUS AFP (NON POLIO) \\
    \midrule
    \emph{1} & \emph{2} & \emph{3} & \emph{4} & \emph{5} \\
    \midrule
	1 & Manggar           & Manggar       & 10.207 & 0 \\
	2 & Damar             & Mengkubang    &  3.324 & 0 \\
	3 & Kelapa Kampit     & Kelapa Kampit &  4.552 & 0 \\
	4 & Gantung           & Gantung       &  8.085 & 0 \\
	5 & Simpang Renggiang & Renggiang     &  1.759 & 0 \\
	6 & Simpang Pesak     & Simpang Pesak &  2.191 & 0 \\
	7 & Dendang           & Dendang       &  2.464 & 0 \\
    \midrule
    \multicolumn{3}{l}{JUMLAH KAB.}       & 32.580 & 0 \\
    \midrule
    \multicolumn{3}{Y{20em}}{AFP RATE (NON POLIO) PER 100.000 PENDUDUK USIA $<$ 15 TAHUN} & & 0,00 \\
    \bottomrule
\end{tabular}%

}

\vspace{2ex}
{\small \textsuperscript{1}\emph{Accute Flaccid Paralysis}}

\vfill
Sumber: Subkoordinator Surveilans, Epidemiologi dan Imunisasi\par 
