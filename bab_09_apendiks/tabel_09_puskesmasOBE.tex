\phantomsection
\addcontentsline{lot}{section}{\protect\numberline{}Tabel 9 - Persentase puskesmas dengan ketersediaan obat dan vaksin esensial}
\ra{1.3}

{\centering
\begin{tabular}{cY{4cm}Y{4cm}Z{4cm}}
    \multicolumn{4}{l}{Tabel 9}\\
    \multicolumn{4}{c}{PERSENTASE PUSKESMAS DENGAN KETERSEDIAAN OBAT VAKSIN ESENSIAL}\\
    \multicolumn{4}{c}{MENURUT KECAMATAN DAN PUSKESMAS}\\
    \multicolumn{4}{c}{KABUPATEN BELITUNG TIMUR}\\
    \multicolumn{4}{c}{TAHUN \tP}\\
    \toprule
    NO & KECAMATAN & PUSKESMAS & KETERSEDIAAN OBAT ESENSIAL \\
    \midrule
    \emph{1} & \emph{2} & \emph{3} & \emph{4}\\
	\midrule
    1 & Manggar            & Manggar       & \checkmark \\
    2 & Damar              & Mengkubang    & \checkmark \\
    3 & Kelapa Kampit      & Kelapa Kampit & \checkmark \\
    4 & Gantung            & Gantung       & \checkmark \\
    5 & Simpang Renggiang  & Renggiang     & \checkmark \\
    6 & Simpang Pesak      & Simpang Pesak & \checkmark \\
    7 & Dendang            & Dendang       & \checkmark \\
    \midrule
    \multicolumn{3}{l}{\parbox{10cm}{JUMLAH PUSKESMAS YANG MEMILIKI 80\% OBAT ESENSIAL}} & 7  \\
    \midrule[0.1pt]
    \multicolumn{3}{l}{JUMLAH PUSKESMAS YANG MELAPOR} & 7  \\
    \midrule[0.1pt]
    \multicolumn{3}{l}{\parbox{10cm}{\% PUSKESMAS DENGAN KETERSEDIAAN OBAT ESENSIAL}} & 100,00\% \\
    \bottomrule
\end{tabular}%

}

\vfill
Sumber: Subkoordinator Kefarmasian dan Alat Kesehatan\par