\phantomsection
\addcontentsline{lot}{section}{\protect\numberline{}Tabel 11 - Persentase puskesmas dengan ketersediaan obat dan vaksin esensial}
\ra{1.3}

{\centering
\begin{tabular}{cY{8cm}Z{4cm}}
	\multicolumn{3}{l}{Tabel 11}\\
	\multicolumn{3}{c}{KETERSEDIAAN VAKSIN IMUNISASI DASAR LENGKAP (IDL)}\\
	\multicolumn{3}{c}{KABUPATEN BELITUNG TIMUR}\\
	\multicolumn{3}{c}{TAHUN \tP}\\
	\toprule
	   NO    & NAMA VAKSIN                                                               & KETERSEDIAAN VAKSIN IDL \\
	\midrule
	\emph{1} & \emph{2}                                                                  & \emph{3}                \\
	\midrule
	   1     & Vaksin Hepatitis B                                                        & \checkmark              \\
	   2     & Vaksin BCG                                                                & \checkmark              \\
	   3     & Vaksin DPT-HB-HIB                                                         & \checkmark              \\
	   4     & Vaksin Polio                                                              & \checkmark              \\
	   5     & Vaksin Campak/Vaksin Campak Rubella (MR)                                  & \checkmark              \\
	\midrule
	\multicolumn{2}{l}{\parbox{10cm}{JUMLAH ITEM VAKSIN IDL YANG TERSEDIA DI KABUPATEN}} & 5                       \\
	\midrule[0.1pt]
	\multicolumn{2}{l}{\parbox{10cm}{\% KABUPATEN DENGAN KETERSEDIAAN VAKSIN IDL}}       & 100,00\%                \\
	\bottomrule
\end{tabular}%

}

\vfill
Sumber: Subkoordinator Kefarmasian dan Alat Kesehatan\par
