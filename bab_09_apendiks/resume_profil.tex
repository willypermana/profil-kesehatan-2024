\phantomsection
\addcontentsline{lot}{chapter}{\protect\numberline{}Resume Profil}
% Table generated by Excel2LaTeX from sheet 'Resume'
\begin{small}
\begin{longtable}{rY{12em}rrrrY{5em}Y{5em}}
    \multicolumn{8}{c}{RESUME PROFIL KESEHATAN}\\
    \multicolumn{8}{c}{KABUPATEN BELITUNG TIMUR}\\
    \multicolumn{8}{c}{TAHUN \tP}\\
	\\ \toprule

	\multirow{2}{*}{\textbf{No}} & \multirow{2}{*}{\textbf{INDIKATOR}}                & \multicolumn{5}{c}{\textbf{ANGKA/ NILAI}} & \multirow{2}{5em}{\raggedright\textbf{No. Lampiran}} \\
    \cmidrule{3-7}
	&                                                                                 & \textbf{L} & \textbf{P} & \textbf{L + P} & \textbf{Jumlah} & \textbf{Satuan} & \\ \midrule
	\endfirsthead
	\\ \toprule
	\multirow{2}{*}{\textbf{No}} & \multirow{2}{*}{\textbf{INDIKATOR}}                & \multicolumn{5}{c}{\textbf{ANGKA/ NILAI}} & \multirow{2}{5em}{\raggedright\textbf{No. Lampiran}} \\
    \cmidrule{3-7}
    &                                                                                 & \textbf{L} & \textbf{P} & \textbf{L + P} & \textbf{Jumlah} & \textbf{Satuan} & \\ \midrule
    \endhead
    % make bottom border on when table continue to next page
    \midrule
    \endfoot
    \endlastfoot
	           \textbf{I} & \textbf{GAMBARAN UMUM}                                                                &        &        &         &                   &                                &          \\
	                    1 & Luas Wilayah                                                                          &        &        &         &             2.507 & Km\textsuperscript{2}          & Tabel 1  \\
	  \rowcolor{black!5}2 & Jumlah Desa/ Kelurahan                                                                &        &        &         &                39 & Desa/ Kelurahan                & Tabel 1  \\
	                    3 & Jumlah Penduduk                                                                       & 66.201 & 62.847 & 129.048 &                   & Jiwa                           & Tabel 2  \\
	 \rowcolor{black!5} 4 & Rata-rata jiwa/ rumah tangga                                                          &        &        &         &              2,89 & Jiwa                           & Tabel 1  \\
	                    5 & Kepadatan Penduduk /Km\textsuperscript{2}                                             &        &        &         &             51,48 & Jiwa/Km\textsuperscript{2}     & Tabel 1  \\
	 \rowcolor{black!5} 6 & Rasio Beban Tanggungan                                                                &        &        &         &             44,31 & per 100 penduduk produktif     & Tabel 2  \\
	                    7 & Rasio Jenis Kelamin                                                                   &        &        &         &            105,34 &                                & Tabel 2  \\
	 \rowcolor{black!5} 8 & Penduduk 15 tahun ke atas melek huruf                                                 &   0,00 &   0,00 &   99,14 &                   & \%                             & Tabel 3  \\
	                    9 & Penduduk 15 tahun yang memiliki ijazah tertinggi                                      &        &        &         &                   &                                &          \\
	                      & a. SMP/ MTs                                                                           &  17,93 &  17,10 &   17,52 &                   & \%                             & Tabel 3  \\
	                      & b. SMA/ MA                                                                            &  26,55 &  22,24 &   24,45 &                   & \%                             & Tabel 3  \\
	                      & c. Sekolah menengah kejuruan                                                          &   8,78 &   4,60 &    6,74 &                   & \%                             & Tabel 3  \\
	                      & d. Diploma I/Diploma II                                                               &   0,40 &   0,56 &    0,48 &                   & \%                             & Tabel 3  \\
	                      & e. Akademi/Diploma III                                                                &   1,22 &   2,68 &    1,93 &                   & \%                             & Tabel 3  \\
	                      & f.  S1/Diploma IV                                                                     &   4,69 &   8,49 &    6,55 &                   & \%                             & Tabel 3  \\
	                      & g. S2/S3 (Master/Doktor)                                                              &   0,27 &   0,16 &    0,21 &                   & \%                             & Tabel 3  \\
	                      &                                                                                       &        &        &         &                   &                                &          \\
	          \textbf{II} & \textbf{SARANA KESEHATAN}                                                             &        &        &         &                   &                                &          \\
	        \textbf{II.1} & \textbf{Sarana Kesehatan}                                                             &        &        &         &                   &                                &          \\
	                   10 & Jumlah Rumah Sakit Umum                                                               &        &        &         &                 1 & RS                             & Tabel 4  \\
	 \rowcolor{black!5}11 & Jumlah Rumah Sakit Khusus                                                             &        &        &         &                 0 & RS                             & Tabel 4  \\
	                   12 & Jumlah Puskesmas Rawat Inap                                                           &        &        &         &                 4 & Puskesmas                      & Tabel 4  \\
	 \rowcolor{black!5}13 & Jumlah Puskesmas non-Rawat Inap                                                       &        &        &         &                 3 & Puskesmas                      & Tabel 4  \\
	                   14 & Jumlah Puskesmas Keliling                                                             &        &        &         &                 0 & Puskesmas keliling             & Tabel 4  \\
	 \rowcolor{black!5}15 & Jumlah Puskesmas pembantu                                                             &        &        &         &                15 & Pustu                          & Tabel 4  \\
	                   16 & Jumlah Apotek                                                                         &        &        &         &                25 & Apotek                         & Tabel 4  \\
	 \rowcolor{black!5}17 & Jumlah Klinik Pratama                                                                 &        &        &         &                 7 & Klinik Pratama                 & Tabel 4  \\
	                   18 & Jumlah Klinik Utama                                                                   &        &        &         &                 1 & Klinik Utama                   & Tabel 4  \\
	 \rowcolor{black!5}19 & RS dengan kemampuan pelayanan gadar level 1                                           &        &        &         &             100,0 & \%                             & Tabel 6  \\
	                      &                                                                                       &        &        &         &                   &                                &          \\
                          &                                                                                       &        &        &         &                   &                                &          \\
                          &                                                                                       &        &        &         &                   &                                &          \\
	        \textbf{II.2} & \textbf{Akses dan Mutu Pelayanan Kesehatan}                                           &        &        &         &                   &                                &          \\
	                   20 & Cakupan Kunjungan Rawat Jalan                                                         & 143,94 & 195,60 &  169,10 &                   & \%                             & Tabel 5  \\
	 \rowcolor{black!5}21 & Cakupan Kunjungan Rawat Inap                                                          &   4,83 &   6,14 &    0,00 &                   & \%                             & Tabel 5  \\
	                   22 & Angka kematian kasar/ \emph{Gross Death Rate} (GDR) di RS                             &  34,84 &  22,52 &   27,53 &                   & per 1.000 pasien keluar        & Tabel 7  \\
	 \rowcolor{black!5}23 & Angka kematian murni/ \emph{Nett Death Rate} (NDR) di RS                              &  38,85 &  15,66 &   25,09 &                   & per 1.000 pasien keluar        & Tabel 7  \\
	                   24 & \emph{Bed Occupation Rate} (BOR) di RS                                                &        &        &         &             50,27 & \%                             & Tabel 8  \\
	 \rowcolor{black!5}25 & \emph{Bed Turn Over (BTO)} di RS                                                      &        &        &         &             52,46 & Kali                           & Tabel 8  \\
	                   26 & \emph{Turn of Interval} (TOI) di RS                                                   &        &        &         &              3,46 & Hari                           & Tabel 8  \\
	 \rowcolor{black!5}27 & \emph{Average Length of Stay (ALOS)} di RS                                            &        &        &         &              3,52 & Hari                           & Tabel 8  \\
	                   28 & Puskesmas dengan ketersediaan obat vaksin \& essensial                                &        &        &         &            100,00 & \%                             & Tabel 9  \\
	 \rowcolor{black!5}29 & Persentase Ketersediaan Obat Essensial                                                &        &        &         &             40,00 & \%                             & Tabel 10 \\
	                   30 & Persentase kabupaten/kota dengan ketersediaan vaksin IDL                              &        &        &         &            100,00 & \%                             & Tabel 11 \\
	                      &                                                                                       &        &        &         &                   &                                &          \\
	                      &                                                                                       &        &        &         &                   &                                &          \\
	        \textbf{II.3} & \textbf{Upaya Kesehatan Bersumberdaya Masyarakat (UKBM)}                              &        &        &         &                   &                                &          \\
	                   31 & Jumlah Posyandu                                                                       &        &        &         &               133 & Posyandu                       & Tabel 12 \\
	 \rowcolor{black!5}32 & Posyandu Aktif                                                                        &        &        &         &            100,00 & \%                             & Tabel 12 \\
	                   33 & Rasio posyandu per 100 balita                                                         &        &        &         &              1,42 & per 100 balita                 & Tabel 12 \\
	 \rowcolor{black!5}34 & Posbindu PTM                                                                          &        &        &         &                58 & Posbindu PTM                   & Tabel 12 \\
	                      &                                                                                       &        &        &         &                   &                                &          \\
	         \textbf{III} & \textbf{SUMBER DAYA MANUSIA KESEHATAN}                                                &        &        &         &                   &                                &          \\
	                   35 & Jumlah Dokter Spesialis                                                               &      8 &      9 &      17 &                   & Orang                          & Tabel 13 \\
	 \rowcolor{black!5}36 & Jumlah Dokter Umum                                                                    &     22 &     28 &      50 &                   & Orang                          & Tabel 13 \\
	                   37 & Rasio Dokter (spesialis+umum)                                                         &        &        &   51,92 &                   & per 100.000 penduduk           & Tabel 13 \\
	 \rowcolor{black!5}38 & Jumlah Dokter Gigi + Dokter Gigi Spesialis                                            &      0 &      9 &       9 &                   & Orang                          & Tabel 13 \\
	                   39 & Rasio Dokter Gigi (termasuk Dokter Gigi Spesialis)                                    &        &        &    6,97 &                   & per 100.000 penduduk           & Tabel 13 \\
	 \rowcolor{black!5}40 & Jumlah Bidan                                                                          &        &    176 &         &                   & Orang                          & Tabel 14 \\
	                   41 & Rasio Bidan per 100.000 penduduk                                                      &        & 136,38 &         &                   & per 100.000 penduduk           & Tabel 14 \\
	 \rowcolor{black!5}42 & Jumlah Perawat                                                                        &    140 &    241 &     381 &                   & Orang                          & Tabel 14 \\
	                   43 & Rasio Perawat per 100.000 penduduk                                                    &        &        &  295,24 &                   & per 100.000 penduduk           & Tabel 14 \\
	 \rowcolor{black!5}44 & Jumlah Tenaga Kesehatan Masyarakat                                                    &      7 &     12 &      19 &                   & Orang                          & Tabel 15 \\
	                   45 & Jumlah Tenaga Kesehatan Lingkungan                                                    &      3 &      7 &      10 &                   & Orang                          & Tabel 15 \\
	 \rowcolor{black!5}46 & Jumlah Tenaga Gizi                                                                    &      2 &     21 &      23 &                   & Orang                          & Tabel 15 \\
	                   47 & Jumlah Ahli Teknologi Laboratorium Medik                                              &      5 &     22 &      27 &                   & Orang                          & Tabel 16 \\
	 \rowcolor{black!5}48 & Jumlah Tenaga Teknik Biomedika Lainnya                                                &      5 &      6 &      11 &                   & Orang                          & Tabel 16 \\
	                   49 & Jumlah Tenaga Keterapian Fisik                                                        &      0 &      7 &       7 &                   & Orang                          & Tabel 16 \\
	 \rowcolor{black!5}50 & Jumlah Tenaga Keteknisian Medis                                                       &      9 &     28 &      37 &                   & Orang                          & Tabel 16 \\
	                   51 & Jumlah Tenaga Teknis Kefarmasian                                                      &      6 &     15 &      21 &                   & Orang                          & Tabel 17 \\
	 \rowcolor{black!5}52 & Jumlah Tenaga Apoteker                                                                &      2 &     13 &      15 &                   & Orang                          & Tabel 17 \\
	                   53 & Jumlah Tenaga Kefarmasian                                                             &      8 &     28 &      36 &                   & Orang                          & Tabel 17 \\
	                      &                                                                                       &        &        &         &                   &                                &          \\
	          \textbf{IV} & \textbf{PEMBIAYAAN KESEHATAN}                                                         &        &        &         &                   &                                &          \\
	                   54 & Peserta Jaminan Pemeliharaan Kesehatan                                                &        &        &         &             98,92 & \%                             & Tabel 19 \\
	 \rowcolor{black!5}55 & Total anggaran kesehatan                                                              &        & \multicolumn{3}{r}{ 202.641.232.727} & Rp                             & Tabel 20 \\
	                   56 & APBD kesehatan terhadap APBD Kabupaten                                                &        &        &         &             19,29 & \%                             & Tabel 20 \\
	 \rowcolor{black!5}57 & Anggaran kesehatan perkapita                                                          &        &     \multicolumn{3}{r}{1.570.277,98} & Rp                             & Tabel 20 \\
	                      & \multicolumn{1}{r|}{}                                                                 &        &        &         &                   &                                &          \\
	           \textbf{V} & \textbf{KESEHATAN KELUARGA}                                                           &        &        &         &                   &                                &          \\
	         \textbf{V.1} & \textbf{Kesehatan Ibu}                                                                &        &        &         &                   &                                &          \\
	                   58 & Jumlah Lahir Hidup                                                                    &    979 &    926 &   1.905 &                   & Orang                          & Tabel 21 \\
	 \rowcolor{black!5}59 & Angka Lahir Mati (dilaporkan)                                                         &   8,11 &   9,63 &    8,84 &                   & per 1.000 Kelahiran Hidup      & Tabel 21 \\
	                   60 & Jumlah Kematian Ibu                                                                   &        &      3 &         &                   & Ibu                            & Tabel 22 \\
	 \rowcolor{black!5}61 & Angka Kematian Ibu (dilaporkan)                                                       &        & 157,48 &         &                   & per 100.000 Kelahiran Hidup    & Tabel 22 \\
	                   62 & Kunjungan Ibu Hamil (K1)                                                              &        &  87,29 &         &                   & \%                             & Tabel 24 \\
	 \rowcolor{black!5}63 & Kunjungan Ibu Hamil (K4)                                                              &        &  85,26 &         &                   & \%                             & Tabel 24 \\
	                   64 & Kunjungan Ibu Hamil (K6)                                                              &        &  83,37 &         &                   & \%                             & Tabel 24 \\
	 \rowcolor{black!5}65 & Persalinan di Fasyankes                                                               &        &  90,04 &         &                   & \%                             & Tabel 24 \\
	                   66 & Pelayanan Ibu Nifas KF Lengkap                                                        &        &  89,28 &         &                   & \%                             & Tabel 24 \\
	 \rowcolor{black!5}67 & Ibu Nifas Mendapat Vitamin A                                                          &        &  89,57 &         &                   & \%                             & Tabel 24 \\
	                   68 & Ibu hamil dengan imunisasi Td2+                                                       &        &  86,62 &         &                   & \%                             & Tabel 25 \\
	 \rowcolor{black!5}69 & Ibu Hamil Mendapat Tablet Tambah Darah 90                                             &        &  82,02 &         &                   & \%                             & Tabel 28 \\
	                   70 & Ibu Hamil Mengonsumsi Tablet Tambah Darah 90                                          &        &  82,02 &         &                   & \%                             & Tabel 28 \\
	 \rowcolor{black!5}71 & Bumil dengan Komplikasi Kebidanan yang Ditangani                                      &        & 112,21 &         &                   & \%                             & Tabel 32 \\
	                   72 & Peserta KB Aktif Modern                                                               &        &        &   79,03 &                   & \%                             & Tabel 29 \\
	 \rowcolor{black!5}73 & Peserta KB Pasca Persalinan                                                           &        &        &   73,98 &                   & \%                             & Tabel 31 \\
	                      &                                                                                       &        &        &         &                   &                                &          \\
	         \textbf{V.2} & \textbf{Kesehatan Anak}                                                               &        &        &         &                   &                                &          \\
	                   74 & Jumlah Kematian Neonatal                                                              &     11 &      3 &      14 &                   & neonatal                       & Tabel 34 \\
	 \rowcolor{black!5}75 & Angka Kematian Neonatal (dilaporkan)                                                  &  11,24 &   3,24 &    7,35 &                   & per 1.000 Kelahiran Hidup      & Tabel 34 \\
	                   76 & Jumlah Bayi Mati                                                                      &     13 &      4 &      17 &                   & bayi                           & Tabel 34 \\
	 \rowcolor{black!5}77 & Angka Kematian Bayi (dilaporkan)                                                      &  13,28 &   4,32 &    8,92 &                   & per 1.000 Kelahiran Hidup      & Tabel 34 \\
	                   78 & Jumlah Balita Mati                                                                    &     14 &      5 &      19 &                   & Balita                         & Tabel 34 \\
	 \rowcolor{black!5}79 & Angka Kematian Balita (dilaporkan)                                                    &  14,30 &   5,40 &    9,97 &                   & per 1.000 Kelahiran Hidup      & Tabel 34 \\
	                   80 & Bayi baru lahir ditimbang                                                             &  92,27 &  45,93 &   94,49 &                   & \%                             & Tabel 37 \\
	 \rowcolor{black!5}81 & Berat Badan Bayi Lahir Rendah (BBLR)                                                  &   7,35 &   7,99 &    7,66 &                   & \%                             & Tabel 37 \\
	                   82 & Kunjungan Neonatus 1 (KN 1)                                                           &  91,33 &  97,49 &   94,25 &                   & \%                             & Tabel 38 \\
	 \rowcolor{black!5}83 & Kunjungan Neonatus 3 kali (KN Lengkap)                                                &  91,89 &  96,96 &   94,30 &                   & \%                             & Tabel 38 \\
	                   84 & Bayi yang diberi ASI Eksklusif                                                        &        &        &   46,96 &                   & \%                             & Tabel 39 \\
	 \rowcolor{black!5}85 & Pelayanan kesehatan bayi                                                              &  92,94 & 100,95 &   96,74 &                   & \%                             & Tabel 40 \\
	                   86 & Desa/ Kelurahan UCI                                                                   &        &        &         &             89,74 & \%                             & Tabel 41 \\
	 \rowcolor{black!5}87 & Cakupan Imunisasi Campak/ Rubela pada Bayi                                            &  88,84 &  94,91 &   91,71 &                   & \%                             & Tabel 43 \\
	                   88 & Imunisasi dasar lengkap pada bayi                                                     &  88,74 &  97,24 &   92,77 &                   & \%                             & Tabel 43 \\
	 \rowcolor{black!5}89 & Bayi Mendapat Vitamin A                                                               &        &        &   86,60 &                   & \%                             & Tabel 45 \\
	                   90 & Anak Balita Mendapat Vitamin A                                                        &        &        &   86,21 &                   & \%                             & Tabel 45 \\
	 \rowcolor{black!5}91 & Balita Mendapatkan Vitamin A                                                          &        &        &   86,60 &                   & \%                             & Tabel 45 \\
	                   92 & Balita Memiliki Buku KIA                                                              &        &        &   70,20 &                   & \%                             & Tabel 46 \\
	 \rowcolor{black!5}93 & Balita Dipantau Pertumbuhan dan Perkembangan                                          &        &        &   99,01 &                   & \%                             & Tabel 46 \\
	                   94 & Balita ditimbang (D/S)                                                                &     40 &     49 &   43,92 &                   & \%                             & Tabel 47 \\
	 \rowcolor{black!5}95 & Balita Berat Badan Kurang (BB/U)                                                      &        &        &    8,57 &                   & \%                             & Tabel 48 \\
	                   96 & Balita pendek (TB/U)                                                                  &        &        &    5,00 &                   & \%                             & Tabel 48 \\
	 \rowcolor{black!5}97 & Balita Gizi Kurang (BB/TB)                                                            &        &        &    3,57 &                   & \%                             & Tabel 48 \\
	                   98 & Balita Gizi Buruk (BB/TB)                                                             &        &        &    0,04 &                   & \%                             & Tabel 48 \\
	 \rowcolor{black!5}99 & Cakupan Penjaringan Kesehatan Siswa Kelas 1 SD/ MI                                    &        &        &  100,00 &                   & \%                             & Tabel 49 \\
	                  100 & Cakupan Penjaringan Kesehatan Siswa Kelas 7 SMP/ MTs                                  &        &        &  100,00 &                   & \%                             & Tabel 49 \\
	\rowcolor{black!5}101 & Cakupan Penjaringan Kesehatan Siswa Kelas 10 SMA/ MA                                  &        &        &  100,00 &                   & \%                             & Tabel 49 \\
	                  102 & Pelayanan kesehatan pada usia pendidikan dasar                                        &        &        &   99,96 &                   & \%                             & Tabel 49 \\
	                      &                                                                                       &        &        &         &                   &                                &          \\
	         \textbf{V.3} & \textbf{Kesehatan Usia Produktif dan Usia Lanjut}                                     &        &        &         &                   &                                &          \\
	                  103 & Pelayanan Kesehatan Usia Produktif                                                    &  64,09 & 114,80 &   88,49 &                   & \%                             & Tabel 52 \\
	\rowcolor{black!5}104 & Catin Mendapatkan Layanan Kesehatan                                                   &  97,84 &  97,84 &   97,84 &                   & \%                             & Tabel 53 \\
	                  105 & Pelayanan Kesehatan Usila (60+ tahun)                                                 &  77,00 & 109,66 &   87,07 &                   & \%                             & Tabel 54 \\
	                      &                                                                                       &        &        &         &                   &                                &          \\
	          \textbf{VI} & \textbf{PENGENDALIAN PENYAKIT}                                                        &        &        &         &                   &                                &          \\
	        \textbf{VI.1} & \textbf{Pengendalian Penyakit Menular Langsung}                                       &        &        &         &                   &                                &          \\
	                  106 & Persentase orang terduga TBC mendapatkan pelayanan kesehatan sesuai standar           &        &        &   65,67 &                   & \%                             & Tabel 56 \\
	\rowcolor{black!5}107 & \emph{Treatment Coverage} TBC                                                         &        &        &   42,28 &                   & \%                             & Tabel 56 \\
	                  108 & Cakupan penemuan kasus TBC anak                                                       &        &        &   72,35 &                   & \%                             & Tabel 56 \\
	\rowcolor{black!5}109 & Angka kesembuhan BTA+                                                                 &  48,21 &  60,00 &   52,75 &                   & \%                             & Tabel 57 \\
	                  110 & Angka pengobatan lengkap semua kasus TBC                                              &  32,73 &  38,89 &   87,91 &                   & \%                             & Tabel 57 \\
	\rowcolor{black!5}111 & Angka keberhasilan pengobatan (\emph{Success Rate}) semua kasus TBC                   &  81,82 &  97,22 &   87,91 &                   & \%                             & Tabel 57 \\
	                  112 & Jumlah kematian selama pengobatan tuberkulosis                                        &        &        &    6,59 &                   & \%                             & Tabel 57 \\
	\rowcolor{black!5}113 & Penemuan penderita pneumonia pada balita                                              &        &        &   10,92 &                   & \%                             & Tabel 58 \\
	                  114 & Puskesmas yang melakukan tatalaksana standar pneumonia min 60\%                       &        &        &         &            100,00 & \%                             & Tabel 58 \\
	\rowcolor{black!5}115 & Jumlah Kasus HIV                                                                      &      0 &      0 &       0 &                   & Kasus                          & Tabel 59 \\
	                  116 & Persentase ODHIV Baru Mendapat Pengobatan ARV                                         &        &        &   73,08 &                   & \%                             & Tabel 60 \\
	\rowcolor{black!5}117 & Persentase Penderita Diare pada Semua Umur Dilayani                                   &        &        &   29,62 &                   & \%                             & Tabel 61 \\
	                  118 & Persentase Penderita Diare pada Balita Dilayani                                       &        &        &   29,62 &                   & \%                             & Tabel 61 \\
	\rowcolor{black!5}119 & Persentase Ibu hamil diperiksa Hepatitis                                              &        &        &   83,33 &                   & \%                             & Tabel 62 \\
	                  120 & Persentase Ibu hamil diperiksa Reaktif Hepatitis                                      &        &        &    2,33 &                   & \%                             & Tabel 62 \\
	\rowcolor{black!5}121 & Persentase Bayi dari Bumil Reakif Hepatitis Diperiksa                                 &        &        &  100,00 &                   & \%                             & Tabel 63 \\
	                  122 & Jumlah Kasus Baru Kusta (PB+MB)                                                       &      4 &      1 &       5 &                   & Kasus                          & Tabel 64 \\
	\rowcolor{black!5}123 & Angka penemuan kasus baru kusta (NCDR)                                                &      6 &      2 &       4 &                   & per 100.000 penduduk           & Tabel 64 \\
	                  124 & Persentase Kasus Baru Kusta anak < 15 Tahun                                           &        &        &    0,00 &                   & \%                             & Tabel 65 \\
	\rowcolor{black!5}125 & Persentase Cacat Tingkat 0 Penderita Kusta                                            &        &        &   80,00 &                   & \%                             & Tabel 65 \\
	                  126 & Persentase Cacat Tingkat 2 Penderita Kusta                                            &        &        &   20,00 &                   & \%                             & Tabel 65 \\
	\rowcolor{black!5}127 & Angka Cacat Tingkat 2 Penderita Kusta                                                 &        &        &    7,75 &                   & per 100.000 penduduk           & Tabel 65 \\
	                  128 & Angka Prevalensi Kusta                                                                &        &        &    0,39 &                   & per 10.000 Penduduk            & Tabel 66 \\
	\rowcolor{black!5}129 & Penderita Kusta PB Selesai Berobat (RFT PB)                                           &        &        &    NULL &                   & \%                             & Tabel 67 \\
	                  130 & Penderita Kusta MB Selesai Berobat (RFT MB)                                           &        &        &  100,00 &                   & \%                             & Tabel 67 \\
	                      &                                                                                       &        &        &         &                   &                                &          \\
	        \textbf{VI.2} & \textbf{Pengendalian Penyakit yang Dapat Dicegah dengan Imunisasi}                    &        &        &         &                   &                                &          \\
	                  131 & AFP \emph{Rate} (non polio) < 15 tahun                                                &        &        &    0,00 &                   & per 100.000 penduduk <15 tahun & Tabel 68 \\
	\rowcolor{black!5}132 & Jumlah kasus difteri                                                                  &      0 &      0 &       0 &                   & Kasus                          & Tabel 69 \\
	                  133 & \emph{Case Fatality Rate} difteri                                                     &        &        &    NULL &                   & \%                             & Tabel 69 \\
	\rowcolor{black!5}134 & Jumlah kasus pertusis                                                                 &      0 &      0 &       0 &                   & Kasus                          & Tabel 69 \\
	                  135 & Jumlah kasus tetanus neonatorum                                                       &      0 &      0 &       0 &                   & Kasus                          & Tabel 69 \\
	\rowcolor{black!5}136 & \emph{Case Fatality Rate} tetanus neonatorum                                          &        &        &    NULL &                   & \%                             & Tabel 69 \\
	                  137 & Jumlah kasus hepatitis B                                                              &      0 &      0 &       0 &                   & Kasus                          & Tabel 69 \\
	\rowcolor{black!5}138 & Jumlah kasus suspek campak                                                            &      0 &      0 &       0 &                   & Kasus                          & Tabel 69 \\
	                  139 & Insiden rate suspek campak                                                            &   0,00 &   0,00 &    0,00 &                   & per 100.000 penduduk           & Tabel 69 \\
	\rowcolor{black!5}140 & KLB ditangani < 24 jam                                                                &        &        &         &              NULL & \%                             & Tabel 70 \\
	                      &                                                                                       &        &        &         &                   &                                &          \\
	                      &                                                                                       &        &        &         &                   &                                &          \\
	                      &                                                                                       &        &        &         &                   &                                &		  \\				  
	                      &                                                                                       &        &        &         &                   &                                &		  \\
	        \textbf{VI.3} & \textbf{Pengendalian Penyakit Tular Vektor dan Zoonotik}                              &        &        &         &                   &                                &          \\
	                  141 & Angka kesakitan (\emph{Incidence Rate})DBD                                            &        &        &   31,77 &                   & per 100.000 penduduk           & Tabel 72 \\
	\rowcolor{black!5}142 & Angka kematian (\emph{Case Fatality Rate}) DBD                                        &   0,00 &   0,00 &    0,00 &                   & \%                             & Tabel 72 \\
	                  143 & Angka kesakitan malaria (\emph{Annual Parasite Incidence})                            &        &        &    0,00 &                   & per 1.000 penduduk             & Tabel 73 \\
	\rowcolor{black!5}144 & Konfirmasi laboratorium pada suspek malaria                                           &        &        &  100,00 &                   & \%                             & Tabel 73 \\
	                  145 & Pengobatan standar kasus malaria positif                                              &        &        &    NULL &                   & \%                             & Tabel 73 \\
	\rowcolor{black!5}146 & \emph{Case Fatality Rate} malaria                                                     &   NULL &   NULL &    NULL &                   & \%                             & Tabel 73 \\
	                  147 & Penderita kronis filariasis                                                           &     10 &      1 &      11 &                   & Kasus                          & Tabel 74 \\
	\rowcolor{black!5}148 & Jumlah Kasus Covid-19                                                                 &        &        &       3 &                   & Kasus                          & Tabel 84 \\
	                  149 & CFR (\emph{Case Fatality Rate}) Covid-19                                              &        &        &    0,00 &                   & \%                             & Tabel 84 \\
	\rowcolor{black!5}150 & Cakupan Total Vaksinasi Covid-19 Dosis 1                                              &        &        &    NULL &                   &                                & Tabel 86 \\
	                  151 & Cakupan Total Vaksinasi Covid-19 Dosis 2                                              &        &        &    NULL &                   &                                & Tabel 87 \\
	                      &                                                                                       &        &        &         &                   &                                &          \\
	        \textbf{VI.4} & \textbf{Pengendalian Penyakit Tidak Menular}                                          &        &        &         &                   &                                &          \\
	                  152 & Penderita Hipertensi Mendapat Pelayanan Kesehatan                                     &  53,71 & 104,23 &   78,37 &                   & \%                             & Tabel 75 \\
	\rowcolor{black!5}153 & Penyandang DM  mendapatkan pelayanan kesehatan sesuai standar                         &        &        &   95,30 &                   & \%                             & Tabel 76 \\
	                  154 & Pemeriksaan IVA pada perempuan usia 30-50 tahun                                       &        &  14,89 &         &                   & \% perempuan usia 30-50 tahun  & Tabel 77 \\
	\rowcolor{black!5}155 & Persentase IVA positif pada perempuan usia 30-50 tahun                                &        &   0,17 &         &                   & \%                             & Tabel 77 \\
	                  156 & Pemeriksaan payudara (SADANIS) pada perempuan 30-50 tahun                             &        &  14,89 &         &                   & \%                             & Tabel 77 \\
	\rowcolor{black!5}157 & Persentase tumor/ benjolan payudara pada perempuan 30-50 tahun                        &        &   0,10 &         &                   & \%                             & Tabel 77 \\
	                  158 & Pelayanan Kesehatan Orang dengan Gangguan Jiwa Berat                                  &        &        &  100,00 &                   & \%                             & Tabel 78 \\
	                      &                                                                                       &        &        &         &                   &                                &          \\
	         \textbf{VII} & \textbf{KESEHATAN LINGKUNGAN}                                                         &        &        &         &                   &                                &          \\
	                  159 & Sarana Air Minum yang DiawasiI/ Diperiksa Kualitas Air Minumnya Sesuai Standar (Aman) &        &        &         &             69,23 & \%                             & Tabel 79 \\
	\rowcolor{black!5}160 & KK Stop BABS (SBS)                                                                    &        &        &         &            100,00 & \%                             & Tabel 80 \\
	                  161 & KK dengan Akses terhadap Fasilitas Sanitasi yang Layak                                &        &        &         &             97,64 & \%                             & Tabel 80 \\
	\rowcolor{black!5}162 & KK dengan Akses terhadap Fasilitas Sanitasi yang Aman                                 &        &        &         &             10,47 & \%                             & Tabel 80 \\
	                  163 & Desa/ Kelurahan Stop BABS (SBS)                                                       &        &        &         &            100,00 & \%                             & Tabel 81 \\
	\rowcolor{black!5}164 & KK Cuci Tangan Pakai Sabun (CTPS)                                                     &        &        &         &             79,24 & \%                             & Tabel 81 \\
	                  165 & KK Pengelolaan Air Minum dan Makanan Rumah Tangga (PAMMRT)                            &        &        &         &             86,40 & \%                             & Tabel 81 \\
	\rowcolor{black!5}166 & KK Pengelolaan Sampah Rumah Tangga (PSRT)                                             &        &        &         &             28,96 & \%                             & Tabel 81 \\
	                  167 & KK Pengelolaan Limbah Cair Rumah Tangga (PLCRT)                                       &        &        &         &             14,49 & \%                             & Tabel 81 \\
	\rowcolor{black!5}168 & Desa/ Kelurahan 5 Pilar STBM                                                          &        &        &         &              0,00 & \%                             & Tabel 81 \\
	                  169 & KK Pengelolaan Kualitas Udara dalam Rumah Tangga (PKURT)                              &        &        &         &             61,04 & \%                             & Tabel 81 \\
	\rowcolor{black!5}170 & KK Akses Rumah Sehat                                                                  &        &        &         &             14,46 & \%                             & Tabel 81 \\
	                  171 & Tempat Fasilitas Umum (TFU) yang Dilakukan Pengawasan Sesuai Standar                  &        &        &         &             97,89 & \%                             & Tabel 82 \\
	\rowcolor{black!5}172 & Tempat Pengelolaan Pangan (TPP) Jasa Boga yang Memenuhi Syarat Kesehatan              &        &        &         &             80,43 & \%                             & Tabel 83 \\ 
	 \bottomrule
\end{longtable}%
\end{small}