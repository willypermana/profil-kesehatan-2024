\phantomsection
\addcontentsline{lot}{chapter}{\protect\numberline{}Resume Profil}
\todo{jumlah kunjungan}
% Table generated by Excel2LaTeX from sheet 'Resume'
\begin{small}
\begin{longtable}{rY{12em}rrrrY{5em}Y{5em}}
    \multicolumn{8}{c}{RESUME PROFIL KESEHATAN}\\
    \multicolumn{8}{c}{KABUPATEN BELITUNG TIMUR}\\
    \multicolumn{8}{c}{TAHUN \tP}\\
	\\ \toprule

	\multirow{2}{*}{\textbf{No}} & \multirow{2}{*}{\textbf{INDIKATOR}}                & \multicolumn{5}{c}{\textbf{ANGKA/ NILAI}} & \multirow{2}{5em}{\raggedright\textbf{No. Lampiran}} \\
    \cmidrule{3-7}
	&                                                                                 & \textbf{L} & \textbf{P} & \textbf{L + P} & \textbf{Jumlah} & \textbf{Satuan} & \\ \midrule
	\endfirsthead
	\\ \toprule
	\multirow{2}{*}{\textbf{No}} & \multirow{2}{*}{\textbf{INDIKATOR}}                & \multicolumn{5}{c}{\textbf{ANGKA/ NILAI}} & \multirow{2}{5em}{\raggedright\textbf{No. Lampiran}} \\
    \cmidrule{3-7}
    &                                                                                 & \textbf{L} & \textbf{P} & \textbf{L + P} & \textbf{Jumlah} & \textbf{Satuan} & \\ \midrule
    \endhead
    % make bottom border on when table continue to next page
    \midrule
    \endfoot
    \endlastfoot
               \textbf{I} & \textbf{GAMBARAN UMUM}                                                                &        &        &         &            &                                     &           \\
	                    1 & Luas Wilayah                                                                          &        &        &         &   2.506,90 & Km\textsuperscript{2}               & Tabel 1   \\
	  \rowcolor{black!5}2 & Jumlah Desa/ Kelurahan                                                                &        &        &         &         39 & Desa/ Kelurahan                     & Tabel 1   \\
	                    3 & Jumlah Penduduk                                                                       & 68.439 & 64.947 & 133.386 &            & Jiwa                                & Tabel 2   \\
	  \rowcolor{black!5}4 & Rata-rata jiwa/rumah tangga                                                           &        &        &         &       3,19 & Jiwa                                & Tabel 1   \\
	                    5 & Kepadatan Penduduk/ Km\textsuperscript{2}                                             &        &        &         &      53,21 & Jiwa/ Km\textsuperscript{2}         & Tabel 1   \\
	  \rowcolor{black!5}6 & Rasio Beban Tanggungan                                                                &        &        &         &      45,70 & per 100 penduduk produktif          & Tabel 2   \\
	                    7 & Rasio Jenis Kelamin                                                                   &        &        &         &     105,38 &                                     & Tabel 2   \\
	  \rowcolor{black!5}8 & Penduduk 15 tahun ke atas melek huruf                                                 &   0,00 &   0,00 &   99,14 &            & \%                                  & Tabel 3   \\
	                    9 & Penduduk 15 tahun yang memiliki ijazah tertinggi                                      &        &        &         &            &                                     &           \\
	                      & a. SMP/ MTs                                                                           &  17,93 &  17,10 &   17,52 &            & \%                                  & Tabel 3   \\
	                      & b. SMA/ MA                                                                            &  26,55 &  22,24 &   24,44 &            & \%                                  & Tabel 3   \\
	                      & c. Sekolah menengah kejuruan                                                          &   8,78 &   4,60 &    6,74 &            & \%                                  & Tabel 3   \\
	                      & d. Diploma I/ Diploma II                                                              &   0,40 &   0,56 &    0,48 &            & \%                                  & Tabel 3   \\
	                      & e. Akademi/ Diploma III                                                               &   1,22 &   2,68 &    1,93 &            & \%                                  & Tabel 3   \\
	                      & f. S1/ Diploma IV                                                                     &   4,69 &   8,49 &    6,55 &            & \%                                  & Tabel 3   \\
	                      & g. S2/ S3 (Master/Doktor)                                                             &   0,27 &   0,16 &    0,21 &            & \%                                  & Tabel 3   \\
	                      & \multicolumn{1}{r}{}                                                                  &        &        &         &            &                                     &           \\
	          \textbf{II} & \textbf{SARANA KESEHATAN}                                                             &        &        &         &            &                                     &           \\
	        \textbf{II.1} & \textbf{Sarana Kesehatan}                                                             &        &        &         &            &                                     &           \\
	                   10 & Jumlah Rumah Sakit Umum                                                               &        &        &         &          1 & RS                                  & Tabel 4   \\
	 \rowcolor{black!5}11 & Jumlah Rumah Sakit Khusus                                                             &        &        &         &          0 & RS                                  & Tabel 4   \\
	                   12 & Jumlah Puskesmas Rawat Inap                                                           &        &        &         &          4 & Puskesmas                           & Tabel 4   \\
	 \rowcolor{black!5}13 & Jumlah Puskesmas non-Rawat Inap                                                       &        &        &         &          3 & Puskesmas                           & Tabel 4   \\
	                   14 & Jumlah Puskesmas Keliling                                                             &        &        &         &          0 & Puskesmas keliling                  & Tabel 4   \\
	 \rowcolor{black!5}15 & Jumlah Puskesmas pembantu                                                             &        &        &         &         15 & Pustu                               & Tabel 4   \\
	                   16 & Jumlah Apotek                                                                         &        &        &         &         27 & Apotek                              & Tabel 4   \\
	 \rowcolor{black!5}17 & Jumlah Klinik Pratama                                                                 &        &        &         &          6 & Klinik Pratama                      & Tabel 4   \\
	                   18 & Jumlah Klinik Utama                                                                   &        &        &         &          1 & Klinik Utama                        & Tabel 4   \\
	 \rowcolor{black!5}19 & RS dengan kemampuan pelayanan gadar level 1                                           &        &        &         &     100,00 & \%                                  & Tabel 6   \\
	                      & \multicolumn{1}{r}{}                                                                  &        &        &         &            &                                     &           \\
	                 II.2 & Akses dan Mutu Pelayanan Kesehatan                                                    &        &        &         &            &                                     &           \\
	                   20 & Cakupan Kunjungan Rawat Jalan                                                         & 105,80 & 126,57 &  115,92 &            & \%                                  & Tabel 5   \\
	 \rowcolor{black!5}21 & Cakupan Kunjungan Rawat Inap                                                          &   2,79 &   4,71 &    3,73 &            & \%                                  & Tabel 5   \\
	                   22 & Angka kematian kasar/ \emph{Gross Death Rate} (GDR) di RS                             &  62,90 &  42,99 &   51,32 &            & per 1.000 pasien keluar             & Tabel 7   \\
	 \rowcolor{black!5}23 & Angka kematian murni/ \emph{Nett Death Rate} (NDR) di RS                              &  34,47 &  22,77 &   27,67 &            & per 1.000 pasien keluar             & Tabel 7   \\
	                   24 & \emph{Bed Occupation Rate} (BOR) di RS                                                &        &        &         &      58,42 & \%                                  & Tabel 8   \\
	 \rowcolor{black!5}25 & \emph{Bed Turn Over (BTO)} di RS                                                      &        &        &         &      61,67 & Kali                                & Tabel 8   \\
	                   26 & \emph{Turn of Interval (TOI)} di RS                                                   &        &        &         &       2,46 & Hari                                & Tabel 8   \\
	 \rowcolor{black!5}27 & \emph{Average Length of Stay (ALOS)} di RS                                            &        &        &         &       3,50 & Hari                                & Tabel 8   \\
	                   28 & Puskesmas dengan ketersediaan obat vaksin \& essensial                                &        &        &         &       1,00 & \%                                  & Tabel 9   \\
	 \rowcolor{black!5}29 & Persentase Ketersediaan Obat Essensial                                                &        &        &         &      38,00 & \%                                  & Tabel 10  \\
	                   30 & Persentase KABUPATEN dengan ketersediaan vaksin IDL                                   &        &        &         &       1,00 & \%                                  & Tabel 11  \\
	                      & \multicolumn{1}{r}{}                                                                  &        &        &         &            &                                     &           \\
	        \textbf{II.3} & \textbf{Upaya Kesehatan Bersumberdaya Masyarakat (UKBM)}                              &        &        &         &            &                                     &           \\
	                   31 & Jumlah Posyandu                                                                       &        &        &         &        133 & Posyandu                            & Tabel 12  \\
	 \rowcolor{black!5}32 & Posyandu Aktif                                                                        &        &        &         &     100,00 & \%                                  & Tabel 12  \\
	                   33 & Rasio posyandu per 100 balita                                                         &        &        &         &       1,26 & per 100 balita                      & Tabel 12  \\
	 \rowcolor{black!5}34 & Posbindu PTM                                                                          &        &        &         &         58 & Posbindu PTM                        & Tabel 12  \\
	                      & \multicolumn{1}{r}{}                                                                  &        &        &         &            &                                     &           \\
	         \textbf{III} & \textbf{SUMBER DAYA MANUSIA KESEHATAN}                                                &        &        &         &            &                                     &           \\
	                   35 & Jumlah Dokter Spesialis                                                               &      6 &      5 &      11 &            & Orang                               & Tabel 13  \\
	 \rowcolor{black!5}36 & Jumlah Dokter Umum                                                                    &     22 &     24 &      46 &            & Orang                               & Tabel 13  \\
	                   37 & Rasio Dokter (spesialis+umum)                                                         &        &        &         &      42,73 & per 100.000 penduduk                & Tabel 13  \\
	 \rowcolor{black!5}38 & Jumlah Dokter Gigi + Dokter Gigi Spesialis                                            &      0 &      8 &       8 &            & Orang                               & Tabel 13  \\
	                   39 & Rasio Dokter Gigi (termasuk Dokter Gigi Spesialis)                                    &        &        &         &       6,00 & per 100.000 penduduk                & Tabel 13  \\
	 \rowcolor{black!5}40 & Jumlah Bidan                                                                          &        &    154 &         &            & Orang                               & Tabel 14  \\
	                   41 & Rasio Bidan per 100.000 penduduk                                                      &        & 115,45 &         &            & per 100.000 penduduk                & Tabel 14  \\
	 \rowcolor{black!5}42 & Jumlah Perawat                                                                        &    135 &    241 &     376 &            & Orang                               & Tabel 14  \\
	                   43 & Rasio Perawat per 100.000 penduduk                                                    &        &        &         &     281,89 & per 100.000 penduduk                & Tabel 14  \\
	 \rowcolor{black!5}44 & Jumlah Tenaga Kesehatan Masyarakat                                                    &     10 &     19 &      29 &            & Orang                               & Tabel 15  \\
	                   45 & Jumlah Tenaga Kesehatan Lingkungan                                                    &      3 &      6 &       9 &            & Orang                               & Tabel 15  \\
	 \rowcolor{black!5}46 & Jumlah Tenaga Gizi                                                                    &      2 &     20 &      22 &            & Orang                               & Tabel 15  \\
	                   47 & Jumlah Ahli Teknologi Laboratorium Medik                                              &      6 &     22 &      28 &            & Orang                               & Tabel 16  \\
	 \rowcolor{black!5}48 & Jumlah Tenaga Teknik Biomedika Lainnya                                                &      5 &      6 &      11 &            & Orang                               & Tabel 16  \\
	                   49 & Jumlah Tenaga Keterapian Fisik                                                        &      0 &      8 &       8 &            & Orang                               & Tabel 16  \\
	 \rowcolor{black!5}50 & Jumlah Tenaga Keteknisian Medis                                                       &      8 &     27 &      35 &            & Orang                               & Tabel 16  \\
	                   51 & Jumlah Tenaga Teknis Kefarmasian                                                      &      6 &     15 &      21 &            & Orang                               & Tabel 17  \\
	 \rowcolor{black!5}52 & Jumlah Tenaga Apoteker                                                                &      2 &     14 &      16 &            & Orang                               & Tabel 17  \\
	                   53 & Jumlah Tenaga Kefarmasian                                                             &      8 &     29 &      37 &            & Orang                               & Tabel 17  \\
	                      & \multicolumn{1}{r}{}                                                                  &        &        &         &            &                                     &           \\
	          \textbf{IV} & \textbf{PEMBIAYAAN KESEHATAN}                                                         &        &        &         &            &                                     &           \\
	                   54 & Peserta Jaminan Pemeliharaan Kesehatan                                                &        &        &         &      97,29 & \%                                  & Tabel 19  \\
	 \rowcolor{black!5}55 & Total anggaran kesehatan                                                              & \multicolumn{4}{r}{217.419.105.329,00} & Rp                                  & Tabel 20  \\
	                   56 & APBD kesehatan terhadap APBD KAB.                                                     &        &        &         &      21,57 & \%                                  & Tabel 20  \\
	 \rowcolor{black!5}57 & Anggaran kesehatan perkapita                                                          &       \multicolumn{4}{r}{1.629.999,44} & Rp                                  & Tabel 20  \\
	                      & \multicolumn{1}{r}{}                                                                  &        &        &         &            &                                     &           \\
	           \textbf{V} & \textbf{KESEHATAN KELUARGA}                                                           &        &        &         &            &                                     &           \\
	         \textbf{V.1} & \textbf{Kesehatan Ibu}                                                                &        &        &         &            &                                     &           \\
	                   58 & Jumlah Lahir Hidup                                                                    &    869 &    803 &   1.672 &            & Orang                               & Tabel 21  \\
	 \rowcolor{black!5}59 & Angka Lahir Mati (dilaporkan)                                                         &  16,97 &   7,42 &   12,40 &            & per 1.000 Kelahiran Hidup           & Tabel 21  \\
	                   60 & Jumlah Kematian Ibu                                                                   &        &      2 &         &            & Ibu                                 & Tabel 21  \\
	 \rowcolor{black!5}61 & Angka Kematian Ibu (dilaporkan)                                                       &        & 119,62 &         &            & per 100.000 Kelahiran Hidup         & Tabel 21  \\
	                   62 & Kunjungan Ibu Hamil (K1)                                                              &        &  76,80 &         &            & \%                                  & Tabel 24  \\
	 \rowcolor{black!5}63 & Kunjungan Ibu Hamil (K4)                                                              &        &  68,94 &         &            & \%                                  & Tabel 24  \\
	                   64 & Kunjungan Ibu Hamil (K6)                                                              &        &  68,94 &         &            & \%                                  & Tabel 24  \\
	 \rowcolor{black!5}65 & Persalinan di Fasyankes                                                               &        &  74,53 &         &            & \%                                  & Tabel 24  \\
	                   66 & Pelayanan Ibu Nifas KF Lengkap                                                        &        &  74,09 &         &            & \%                                  & Tabel 24  \\
	 \rowcolor{black!5}67 & Ibu Nifas Mendapat Vitamin A                                                          &        &  74,53 &         &            & \%                                  & Tabel 24  \\
	                   68 & Ibu hamil dengan imunisasi Td2+                                                       &        &  71,30 &         &            & \%                                  & Tabel 24  \\
	 \rowcolor{black!5}69 & Ibu Hamil Mendapat Tablet Tambah Darah 90                                             &        &  68,54 &         &            & \%                                  & Tabel 28  \\
	                   70 & Ibu Hamil Mengonsumsi Tablet Tambah Darah 90                                          &        &  68,54 &         &            & \%                                  & Tabel 28  \\
	 \rowcolor{black!5}71 & Bumil dengan Komplikasi Kebidanan yang Ditangani                                      &        & 122,23 &         &            & \%                                  & Tabel 32  \\
	                   72 & Peserta KB Aktif Modern                                                               &        &        &   77,68 &            & \%                                  & Tabel 29  \\
	 \rowcolor{black!5}73 & Peserta KB Pasca Persalinan                                                           &        &        &   65,79 &            & \%                                  & Tabel 31  \\
	                      & \multicolumn{1}{r}{}                                                                  &        &        &         &            &                                     &           \\
	         \textbf{V.2} & \textbf{Kesehatan Anak}                                                               &        &        &         &            &                                     &           \\
	                   74 & Jumlah Kematian Neonatal                                                              &     14 &     13 &      27 &            & neonatal                            & Tabel 34  \\
	 \rowcolor{black!5}75 & Angka Kematian Neonatal (dilaporkan)                                                  &  16,11 &  16,19 &   16,15 &            & per 1.000 Kelahiran Hidup           & Tabel 34  \\
	                   76 & Jumlah Bayi Mati                                                                      &     20 &     18 &      38 &            & bayi                                & Tabel 34  \\
	 \rowcolor{black!5}77 & Angka Kematian Bayi (dilaporkan)                                                      &  23,01 &   22,4 &    22,7 &            & per 1.000 Kelahiran Hidup           & Tabel 34  \\
	                   78 & Jumlah Balita Mati                                                                    &     20 &     18 &      38 &            & Balita                              & Tabel 34  \\
	 \rowcolor{black!5}79 & Angka Kematian Balita (dilaporkan)                                                    &  23,01 &  22,42 &   22,73 &            & per 1.000 Kelahiran Hidup           & Tabel 34  \\
	                   80 & Bayi baru lahir ditimbang                                                             & 100,00 & 100,00 &  100,00 &            & \%                                  & Tabel 33  \\
	 \rowcolor{black!5}81 & Berat Badan Bayi Lahir Rendah (BBLR)                                                  &   7,94 &   8,72 &    8,31 &            & \%                                  & Tabel 33  \\
	                   82 & Kunjungan Neonatus 1 (KN 1)                                                           & 100,00 & 100,00 &  100,00 &            & \%                                  & Tabel 38  \\
	 \rowcolor{black!5}83 & Kunjungan Neonatus 3 kali (KN Lengkap)                                                &  99,19 &  98,88 &   99,04 &            & \%                                  & Tabel 38  \\
	                   84 & Bayi yang diberi ASI Eksklusif                                                        &        &        &   37,16 &            & \%                                  & Tabel 39  \\
	 \rowcolor{black!5}85 & Pelayanan kesehatan bayi                                                              &  77,09 &  86,14 &   81,37 &            & \%                                  & Tabel 36  \\
	                   86 & Desa/ Kelurahan UCI                                                                   &        &        &         &      74,36 & \%                                  & Tabel 41  \\
	 \rowcolor{black!5}87 & Cakupan Imunisasi Campak/ Rubela pada Bayi                                            &  79,38 &  51,95 &   84,89 &            & \%                                  & Tabel 43  \\
	                   88 & Imunisasi dasar lengkap pada bayi                                                     &  79,38 &  50,09 &   84,31 &            & \%                                  & Tabel 43  \\
	 \rowcolor{black!5}89 & Bayi Mendapat Vitamin A                                                               &        &        &   97,46 &            & \%                                  & Tabel 45  \\
	                   90 & Anak Balita Mendapat Vitamin A                                                        &        &        &   97,06 &            & \%                                  & Tabel 45  \\
	 \rowcolor{black!5}91 & Balita Mendapatkan Vitamin A                                                          &        &        &   97,46 &            & \%                                  & Tabel 45  \\
	                   92 & Balita Memiliki Buku KIA                                                              &        &        &   98,99 &            & \%                                  & Tabel 46  \\
	 \rowcolor{black!5}93 & Balita Dipantau Pertumbuhan dan Perkembangan                                          &        &        &   90,37 &            & \%                                  & Tabel 46  \\
	                   94 & Balita ditimbang (D/S)                                                                &  61,56 &  59,80 &   60,71 &            & \%                                  & Tabel 47  \\
	 \rowcolor{black!5}95 & Balita Berat Badan Kurang (BB/U)                                                      &        &        &    8,21 &            & \%                                  & Tabel 48  \\
	                   96 & Balita pendek (TB/U)                                                                  &        &        &    4,61 &            & \%                                  & Tabel 48  \\
	 \rowcolor{black!5}97 & Balita Gizi Kurang (BB/TB)                                                            &        &        &    3,18 &            & \%                                  & Tabel 48  \\
	                   98 & Balita Gizi Buruk (BB/TB)                                                             &        &        &    0,01 &            & \%                                  & Tabel 48  \\
	 \rowcolor{black!5}99 & Cakupan Penjaringan Kesehatan Siswa Kelas 1 SD/MI                                     &        &        &  100,00 &            & \%                                  & Tabel 49  \\
	                  100 & Cakupan Penjaringan Kesehatan Siswa Kelas 7 SMP/MTs                                   &        &        &  100,00 &            & \%                                  & Tabel 49  \\
	\rowcolor{black!5}101 & Cakupan Penjaringan Kesehatan Siswa Kelas 10 SMA/MA                                   &        &        &  100,00 &            & \%                                  & Tabel 49  \\
	                  102 & Pelayanan kesehatan pada usia pendidikan dasar                                        &        &        &  100,00 &            & \%                                  & Tabel 49  \\
	                      & \multicolumn{1}{r}{}                                                                  &        &        &         &            &                                     &           \\
	         \textbf{V.3} & \textbf{Kesehatan Usia Produktif dan Usia Lanjut}                                     &        &        &         &            &                                     &           \\
	                  103 & Pelayanan Kesehatan Usia Produktif                                                    &  66,58 & 130,15 &   97,17 &            & \%                                  & Tabel 52  \\
	\rowcolor{black!5}104 & Catin Mendapatkan Layanan Kesehatan                                                   & 100,00 & 100,00 &  100,00 &            & \%                                  & Tabel 53  \\
	                  105 & Pelayanan Kesehatan Usila (60+ tahun)                                                 &  66,70 &  95,80 &   82,09 &            & \%                                  & Tabel 54  \\
	                      & \multicolumn{1}{r}{}                                                                  &        &        &         &            &                                     &           \\
	          \textbf{VI} & \textbf{PENGENDALIAN PENYAKIT}                                                        &        &        &         &            &                                     &           \\
	        \textbf{VI.1} & \textbf{Pengendalian Penyakit Menular Langsung}                                       &        &        &         &            &                                     &           \\
	                  106 & Persentase orang terduga TBC mendapatkan pelayanan kesehatan sesuai standar           &        &        &  114,37 &            & \%                                  & Tabel 56  \\
	\rowcolor{black!5}107 & \emph{Treatment Coverage} TBC                                                         &        &        &   65,87 &            & \%                                  & Tabel 56  \\
	                  108 & Cakupan penemuan kasus TBC anak                                                       &        &        &   91,49 &            & \%                                  & Tabel 56  \\
	\rowcolor{black!5}109 & Angka kesembuhan BTA+                                                                 &  45,21 &  47,06 &   45,56 &            & \%                                  & Tabel 57  \\
	                  110 & Angka pengobatan lengkap semua kasus TBC                                              &  55,83 &  74,55 &   61,71 &            & \%                                  & Tabel 57  \\
	\rowcolor{black!5}111 & Angka keberhasilan pengobatan (\emph{Success Rate}) semua kasus TBC                   &  83,33 &  89,09 &   61,71 &            & \%                                  & Tabel 57  \\
	                  112 & Jumlah kematian selama pengobatan tuberkulosis                                        &        &        &    4,57 &            & \%                                  & Tabel 57  \\
	\rowcolor{black!5}113 & Penemuan penderita pneumonia pada balita                                              &        &        &   11,10 &            & \%                                  & Tabel 58  \\
	                  114 & Puskesmas yang melakukan tatalaksana standar pneumonia min 60\%                       &        &        &         &     100,00 & \%                                  & Tabel 58  \\
	\rowcolor{black!5}115 & Jumlah Kasus HIV                                                                      &      0 &      0 &       0 &            & Kasus                               & Tabel 59  \\
	                  116 & Persentase ODHIV Baru Mendapat Pengobatan ARV                                         &        &        &   82,61 &            & \%                                  & Tabel 60  \\
	\rowcolor{black!5}117 & Persentase Penderita Diare pada Semua Umur Dilayani                                   &        &        &   51,17 &            & \%                                  & Tabel 61  \\
	                  118 & Persentase Penderita Diare pada Balita Dilayani                                       &        &        &   51,17 &            & \%                                  & Tabel 61  \\
	\rowcolor{black!5}119 & Persentase Ibu hamil diperiksa Hepatitis                                              &        &        &   76,80 &            & \%                                  & Tabel 62  \\
	                  120 & Persentase Ibu hamil diperiksa Reaktif Hepatitis                                      &        &        &    2,72 &            & \%                                  & Tabel 62  \\
	\rowcolor{black!5}121 & Persentase Bayi dari Bumil Reakif Hepatitis Diperiksa                                 &        &        &  100,00 &            & \%                                  & Tabel 62  \\
	                  122 & Jumlah Kasus Baru Kusta (PB+MB)                                                       &      3 &      3 &       6 &            & Kasus                               & Tabel 64  \\
	\rowcolor{black!5}123 & Angka penemuan kasus baru kusta (NCDR)                                                &   4,38 &   4,62 &    4,50 &            & per 100.000 penduduk                & Tabel 64  \\
	                  124 & Persentase Kasus Baru Kusta anak < 15 Tahun                                           &        &        &    0,00 &            & \%                                  & Tabel 64  \\
	\rowcolor{black!5}125 & Persentase Cacat Tingkat 0 Penderita Kusta                                            &        &        &   66,67 &            & \%                                  & Tabel 64  \\
	                  126 & Persentase Cacat Tingkat 2 Penderita Kusta                                            &        &        &   33,33 &            & \%                                  & Tabel 64  \\
	\rowcolor{black!5}127 & Angka Cacat Tingkat 2 Penderita Kusta                                                 &        &        &   14,99 &            & per 100.000 penduduk                & Tabel 64  \\
	                  128 & Angka Prevalensi Kusta                                                                &        &        &    0,45 &            & per 10.000 Penduduk                 & Tabel 65  \\
	\rowcolor{black!5}129 & Penderita Kusta PB Selesai Berobat (RFT PB)                                           &        &        &    NULL &            & \%                                  & Tabel 67  \\
	                  130 & Penderita Kusta MB Selesai Berobat (RFT MB)                                           &        &        &  100,00 &            & \%                                  & Tabel 67  \\
	                      & \multicolumn{1}{r}{}                                                                  &        &        &         &            &                                     &           \\
	        \textbf{VI.2} & \textbf{Pengendalian Penyakit yang Dapat Dicegah dengan Imunisasi}                    &        &        &         &            &                                     &           \\
	                  131 & AFP Rate (non polio) < 15 tahun                                                       &        &        &    0,00 &            & per 100.000 penduduk <15 tahun      & Tabel 68  \\
	\rowcolor{black!5}132 & Jumlah kasus difteri                                                                  &      0 &      0 &       0 &            & Kasus                               & Tabel 69  \\
	                  133 & \emph{Case Fatality Rate} difteri                                                     &        &        &    NULL &            & \%                                  & Tabel 69  \\
	\rowcolor{black!5}134 & Jumlah kasus pertusis                                                                 &      0 &      0 &       0 &            & Kasus                               & Tabel 69  \\
	                  135 & Jumlah kasus tetanus neonatorum                                                       &      0 &      0 &       0 &            & Kasus                               & Tabel 69  \\
	\rowcolor{black!5}136 & \emph{Case Fatality Rate} tetanus neonatorum                                          &        &        &    NULL &            & \%                                  & Tabel 69  \\
	                  137 & Jumlah kasus hepatitis B                                                              &      0 &      0 &       0 &            & Kasus                               & Tabel 69  \\
	\rowcolor{black!5}138 & Jumlah kasus suspek campak                                                            &      0 &      1 &       1 &            & Kasus                               & Tabel 69  \\
	                  139 & \emph{Insidence Rate} suspek campak                                                   &   0,00 &   0,75 &    0,75 &            & per 100.000 penduduk                & Tabel 69  \\
	\rowcolor{black!5}140 & KLB ditangani < 24 jam                                                                &        &        &         &      100,0 & \%                                  & Tabel 63  \\
	                      & \multicolumn{1}{r}{}                                                                  &        &        &         &            &                                     &           \\
	        \textbf{VI.3} & \textbf{Pengendalian Penyakit Tular Vektor dan Zoonotik}                              &        &        &         &            &                                     &           \\
	                  141 & Angka kesakitan (\emph{Incidence Rate})DBD                                            &        &        &   83,22 &            & per 100.000 penduduk                & Tabel 65  \\
	\rowcolor{black!5}142 & Angka kematian (\emph{Case Fatality Rate}) DBD                                        &   2,17 &   0,00 &    0,90 &            & \%                                  & Tabel 65  \\
	                  143 & Angka kesakitan malaria (\emph{Annual Parasit Incidence})                             &        &        &    0,00 &            & per 1.000 penduduk                  & Tabel 66  \\
	\rowcolor{black!5}144 & Konfirmasi laboratorium pada suspek malaria                                           &        &        &  100,00 &            & \%                                  & Tabel 66  \\
	                  145 & Pengobatan standar kasus malaria positif                                              &        &        &    NULL &            & \%                                  & Tabel 66  \\
	\rowcolor{black!5}146 & \emph{Case Fatality Rate} malaria                                                     &   NULL &   NULL &    NULL &            & \%                                  & Tabel 66  \\
	                  147 & Penderita kronis filariasis                                                           &      8 &      1 &       9 &            & Kasus                               & Tabel 67  \\
	                      & \multicolumn{1}{r}{}                                                                  &        &        &         &            &                                     &           \\
	        \textbf{VI.4} & \textbf{Pengendalian Penyakit Tidak Menular}                                          &        &        &         &            &                                     &           \\
	                  148 & Penderita Hipertensi Mendapat Pelayanan Kesehatan                                     &  61,54 & 144,10 &  100,07 &            & \%                                  & Tabel 68  \\
	\rowcolor{black!5}149 & Penyandang DM  mendapatkan pelayanan kesehatan sesuai standar                         &        &        &   88,83 &            & \%                                  & Tabel 69  \\
	                  150 & Pemeriksaan IVA pada perempuan usia 30-50 tahun                                       &        &  12,59 &         &            & \% perempuan usia 30-50 tahun       & Tabel 70  \\
	\rowcolor{black!5}151 & Persentase IVA positif pada perempuan usia 30-50 tahun                                &        &   0,00 &         &            & \%                                  & Tabel 70  \\
	                  152 & Pemeriksaan payudara (SADANIS) pada perempuan 30-50 tahun                             &        &  12,59 &         &            & \%                                  & Tabel 77  \\
	\rowcolor{black!5}153 & Persentase tumor/ benjolan payudara pada perempuan 30-50 tahun                        &        &   0,00 &         &            & \%                                  & Tabel 77  \\
	                  154 & Pelayanan Kesehatan Orang dengan Gangguan Jiwa Berat                                  &        &        &  100,00 &            & \%                                  & Tabel 71  \\
	\rowcolor{black!5}155 & 10 Penyakit Terbanyak Pada Pasien Rawat Jalan                                         &        &        &   1.208 &            & Jumlah kunjungan pasien rawat jalan & Tabel 79a \\
	                  156 & 10 Penyakit Terbanyak Pada Pasien Rawat Inap                                          &        &        &   4.545 &            & Jumlah pasien rawat inap            & Tabel 79b \\
	\rowcolor{black!5}157 & 10 Penyakit Dengan Fatalitas Terbesar Pada Pasien Rawat Inap                          &        &        &    3,32 &            & \%                                  & Tabel 79c \\
	                      & \multicolumn{1}{r}{}                                                                  &        &        &         &            &                                     &           \\
	         \textbf{VII} & \textbf{KESEHATAN LINGKUNGAN}                                                         &        &        &         &            &                                     &           \\
	                  158 & Sarana Air Minum yang DiawasiI/ Diperiksa Kualitas Air Minumnya Sesuai Standar (Aman) &        &        &         &      90,48 & \%                                  & Tabel 80  \\
	\rowcolor{black!5}159 & KK dengan Akses terhadap Fasilitas Sanitasi                                           &        &        &         &     100,00 & \%                                  & Tabel 81  \\
	                  160 & KK Stop BABS (SBS)                                                                    &        &        &         &     100,00 & \%                                  & Tabel 82  \\
	\rowcolor{black!5}161 & KK Cuci Tangan Pakai Sabun (CTPS)                                                     &        &        &         &      82,64 & \%                                  & Tabel 82  \\
	                  162 & KK Pengelolaan Air Minum dan Makanan Rumah Tangga (PAMMRT)                            &        &        &         &      88,45 & \%                                  & Tabel 82  \\
	\rowcolor{black!5}163 & KK Pengelolaan Sampah Rumah Tangga (PSRT)                                             &        &        &         &      38,43 & \%                                  & Tabel 82  \\
	                  164 & KK Pengelolaan Limbah Cair Rumah Tangga (PLCRT)                                       &        &        &         &      22,79 & \%                                  & Tabel 82  \\
	\rowcolor{black!5}165 & Desa/ Kelurahan 5 Pilar STBM                                                          &        &        &         &       7,69 & \%                                  & Tabel 82  \\
	                  166 & Tempat Fasilitas Umum (TFU)  yang dilakukan pengawasan sesuai standar                 &        &        &         &      96,48 & \%                                  & Tabel 83  \\
	\rowcolor{black!5}167 & Tempat Pengelolaan Pangan (TPP) yang memenuhi syarat kesehatan                        &        &        &         &      53,66 & \%                                  & Tabel 84
\end{longtable}%
\end{small}

% Table generated by Excel2LaTeX from sheet 'Resume (unformat)'
\begin{tabular}{rp{31.93em}rrrrll}
	           
\end{tabular}%
