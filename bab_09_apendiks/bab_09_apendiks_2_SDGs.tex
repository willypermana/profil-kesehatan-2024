% Table generated by Excel2LaTeX from sheet 'Sheet1'
\chapter{Sustainable Development Goals (SDGs)}
\begin{center}
    \renewcommand*{\arraystretch}{1.3}
    \begin{longtable}{rY{12em}rrrY{6em}}
        % prevent longtable show up twice in TOC. set caption for firsthead, then set empty starred
        % caption for next heads. in longtable \toprule should be prepended by \\
        \caption{Capaian \emph{Sustainable Development Goals} (SDGs) Bidang Kesehatan Kab. Belitung Timur Tahun \tP}
        \\ \toprule
        No & Target/ Indikator SDGs & Pembilang & Penyebut & Capaian \tP & Satuan\\
        \midrule
        \endfirsthead
        \caption*{}
        \\ \toprule
        No & Target/ Indikator SDGs & Pembilang & Penyebut & Capaian \tP & Satuan\\
        \midrule
        \endhead
        % make bottom border on when table continue to next page
        \hline
        \endfoot
        \endlastfoot
                                     1  & Proporsi peserta jaminan kesehatan melalui SSJN Bidang Kesehatan                                           					 & 129.776 & 133.386 &  97,29 & \%                  \\
                   \rowcolor{black!5}2  & Persentase perempuan pernah kawin umur 15-49 tahun yang proses melahirkan terakhirnya di fasilitas kesehatan                   &   1.680 &   2.254 &  74,53 & \%                  \\
                                     3  & Persentase anak umur 12-23 bulan yang menerima imunisasi dasar lengkap                                        				 &   1.747 &   2.072 &  84,31 & \%                  \\
                   \rowcolor{black!5}4  & Prevalensi penggunaan metode kontrasepsi (CPR) semua cara pada Pasangan Usia Subur (PUS) usia 15-49 tahun yang berstatus kawin &  17.650 &  22.720 &  77,68 & \%                  \\
                                     5  & Prevalensi kekurangan gizi (underweight) pada anak balita                                                   					 &     618 &   7.524 &   8,21 & \%                  \\
                   \rowcolor{black!5}6  & Proporsi penduduk dengan asupan kalori minimum 1400 kkal/kapita/hari                                                           &         &         &    N/A & \%                  \\
                                     7  & Prevalensi stunting (pendek dan sangat pendek) pada anak di bawah lima tahun/ balita                        					 &     347 &   7.523 &   4,61 & \%                  \\
                   \rowcolor{black!5}8  & Prevalensi stunting (pendek dan sangat pendek) pada anak di bawah dua tahun/ baduta                                            &         &         &    N/A & \%                  \\
                                     9  & Prevalensi malnutrisi (berat badan/ tinggi badan) pada anak kurang dari lima tahun berdasarkan tipe         					 &     239 &   7.202 &   3,32 & \%                  \\
                   \rowcolor{black!5}10 & Prevalensi anemia pada ibu hamil                                                                                               &         &         &    N/A & \%                  \\
                                     11 & Persentase bayi usia kurang dari 6 bulan yang mendapatkan ASI eksklusif                                             			 &     770 &   2.072 &  37,16 & \%                  \\
                   \rowcolor{black!5}12 & Persentase perempuan pernah kawin umur 15-49 tahun yang proses melahirkan terakhirnya ditolong oleh tenaga kesehatan terlatih  &   1.680 &   2.254 &  74,53 & \%                  \\
                                     13 & Angka Kematian Balita (AKBa) per 1.000 kelahiran hidup                                                       					 &      38 &   1.672 &  22,73 & /1.000KH            \\
                   \rowcolor{black!5}14 & Angka Kematian Neonatal (AKN) per 1.000 kelahiran hidup                                                                        &      27 &   1.672 &  16,15 & /1.000KH            \\
                                     15 & Angka Kematian Bayi (AKB) per 1.000 kelahiran hidup                                                          					 &      38 &   1.672 &  22,73 & /1.000KH            \\
                   \rowcolor{black!5}16 & Persentase kabupaten/ kota yang mencapai 80\% imunisasi dasar lengkap pada bayi                                                &       1 &       1 & 100,00 & \%                  \\
                                     17 & Prevalensi HIV/AIDS pada populasi dewasa                                                                     					 &      23 & 100.806 &   0,02 & \%                  \\
                   \rowcolor{black!5}18 & Insiden Tuberkulosis (TB) per 100.000 penduduk                                                                                 &     276 & 133.386 & 206,92 & /100.000            \\
                                     19 & Kejadian Malaria per 1.000 orang                                                                           					 &       0 & 133.386 &   0,00 & /1.000              \\
                   \rowcolor{black!5}20 & Jumlah kabupaten/ kota yang mencapai eliminasi malaria                                                                         &         &         &      1 & Kab.                \\
                                     21 & Persentase kabupaten/ kota yang memerlukan deteksi dini untuk infeksi Hepatitis B                         					 &       1 &       1 & 100,00 & \%                  \\
                   \rowcolor{black!5}22 & Jumlah orang yang memerlukan intervensi terhadap penyakit tropis yang terabaikan (Filariasis dan Kusta)                        &         &         &     15 & orang               \\
                                     23 & Jumlah orang yang memerlukan intervensi terhadap penyakit tropis yang terabaikan (Kusta)                     					 &         &         &      6 & orang               \\
                   \rowcolor{black!5}24 & Jumlah provinsi dengan eliminasi kusta                                                                                         &         &         &      0 & Kab.                \\
                                     25 & Jumlah kabupaten/ kota dengan eliminiasi filariasis (berhasil lolos dalam survei penilaian transmisi tahap II) 				 &         &         &      0 & Kab.                \\
                   \rowcolor{black!5}26 & Prevalensi tekanan darah tinggi                                                                                                &         &         &  24,07 & \%                  \\
                                     27 & Prevalensi obesitas pada penduduk umur >=18 tahun                                                           					 &         &         &    N/A & \%                  \\
                   \rowcolor{black!5}28 & Jumlah kabupaten/ kota yang memiliki puskesmas yang menyelenggarakan upaya kesehatan jiwa                                      &         &         &      1 & Kab.                \\
                                     29 & Angka penggunaan metode kontrasepsi jangka panjang (MKJP) cara modern                                       					 &   3.775 &  22.720 &  16,62 & \%                  \\
                   \rowcolor{black!5}30 & Angka kelahiran pada perempuan umur 15-19 tahun (ASFR)                                                                         &         &         &  31,46 & per 1.000 perempuan \\
                                     31 & Total Fertility Rate (TFR)                                                                                   					 &         &         &   1,71 & per perempuan       \\
                   \rowcolor{black!5}32 & Unmet need pelayanan kesehatan                                                                                                 &         &         &    N/A & \%                  \\
                                     33 & Jumlah penduduk yang dicakup asuransi kesehatan atau sistem kesehatan masyarakat per 1.000 penduduk        					 & 129.776 & 133.386 & 972,94 & /1.000 pddk         \\
                   \rowcolor{black!5}34 & Cakupan Jaminan Kesehatan Nasional (JKN)                                                                                       & 129.776 & 133.386 &  97,29 & \%                  \\
                                     35 & Persentase ketersediaan obat dan vaksin di Puskesmas                                                       					 &       7 &       7 & 100,00 & \%                  \\
                   \rowcolor{black!5}36 & Unmet need KB (kebutuhan Keluarga Berencana/ KB yang tidak terpenuhi)                                                          &         &         &    N/A & \%                  \\
                                     37 & Pengetahuan dan pemahaman Pasangan Usia Subur (PUS) tentang metode kontrasepsi modern                       					 &         &         &    N/A & \%                  \\
                   \rowcolor{black!5}38 & Jumlah desa/ kelurahan yang melaksanakan Sanitasi Total Berbasis Masyarakat (STBM)                                             &         &         &      3 & Desa                \\
                                     39 & Jumlah desa/ kelurahan yang Open Defecation Free (ODF)/ Stop Buang Air Besar Sembarangan (SBS)               					 &         &         &     39 & Desa                \\
\bottomrule
\end{longtable}
\par\end{center}