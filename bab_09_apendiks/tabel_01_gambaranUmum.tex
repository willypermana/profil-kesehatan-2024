\phantomsection
% add \protect\numberline{} to get nice indentation in the list of tables
\addcontentsline{lot}{section}{\protect\numberline{}Tabel 1 - Luas wilayah, jumlah desa/ kelurahan, jumlah penduduk, jumlah rumah tangga dan kepadatan penduduk}
\label{tabel-01}
\ra{1.3}
% Table generated by Excel2LaTeX from sheet '1'
{\centering
\begin{tabular}{clrrrrrrrr}
    \multicolumn{10}{l}{Tabel 1}\\
    \multicolumn{10}{c}{LUAS WILAYAH, JUMLAH DESA/ KELURAHAN, JUMLAH PENDUDUK, JUMLAH RUMAH TANGGA,}\\
    \multicolumn{10}{c}{DAN KEPADATAN PENDUDUK MENURUT KECAMATAN.}\\
    \multicolumn{10}{c}{\namaKabupatenKapital}\\
    \multicolumn{10}{c}{TAHUN \tP}\\
    \toprule
    \hrulefill
    \multirow{2}[0]{*}{NO} & \multirow{2}[0]{*}{KECAMATAN} & \multirow{2}[0]{*}{\parbox{6em}{\raggedleft LUAS WILAYAH (Km\textsuperscript{2})}} & \multicolumn{3}{X{16em}}{JUMLAH} & \multirow{2}[0]{*}{\parbox{6em}{\raggedleft JUMLAH PENDUDUK}} & \multirow{2}[0]{*}{\parbox{6em}{\raggedleft JUMLAH RUMAH TANGGA }} & \multirow{2}[0]{*}{\parbox[r]{6em}{\raggedleft RATA-RATA JIWA/ RUMAH TANGGA }} & \multirow{2}[0]{*}{\parbox{6em}{\raggedleft KEPADATAN PENDUDUK PER Km\textsuperscript{2}}} \\
%    \cmidrule(l{2pt}r{2pt}){4-4}\cmidrule(l{2pt}r{2pt}){5-5}\cmidrule(l{2pt}r{2pt}){6-6}
	\cmidrule{4-6}
    & & & DESA & KELURAHAN & \multicolumn{1}{Z{6em}}{DESA + KELURAHAN} & & & & \\
    \midrule
    \emph{1} & \emph{2} & \emph{3} & \emph{4} & \emph{5} & \emph{6} & \emph{7} & \emph{8} & \emph{9} & \emph{10}\\
    \midrule
	1 & Manggar           &  229,00 &  9 & 0 &  9 &  39.982 & 13.769 & 2,90 & 174,59 \\
    2 & Damar             &  236,90 &  5 & 0 &  5 &  13.423 &  4.684 & 2,87 &  56,66 \\
    3 & Kelapa Kampit     &  498,50 &  6 & 0 &  6 &  19.083 &  6.734 & 2,83 &  38,28 \\
    4 & Gantung           &  546,30 &  7 & 0 &  7 &  29.469 & 10.021 & 2,94 &  53,94 \\
    5 & Simpang Renggiang &  390,70 &  4 & 0 &  4 &   7.664 &  2.793 & 2,74 &  19,62 \\
    6 & Simpang Pesak     &  362,20 &  4 & 0 &  4 &   8.644 &  2.946 & 2,93 &  23,87 \\
    7 & Dendang           &  243,30 &  4 & 0 &  4 &  10.783 &  3.650 & 2,95 &  44,32 \\
    \midrule
    \multicolumn{2}{l}{JUMLAH KAB.}& 2.506,90 & 39 & 0 & 39 & 129.048 & 44.597 & 2,89 &  51,48 \\
    \bottomrule
\end{tabular}%

}

\vfill
Sumber: \\
- Dinas Kependudukan dan Pencatatan Sipil \namaKabupaten \\
- Proyeksi internal berdasar data Dinas Kependudukan dan Pencatatan Sipil Kabupaten Belitung Timur tahun 2022 \par
