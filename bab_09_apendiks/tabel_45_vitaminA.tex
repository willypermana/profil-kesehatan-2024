\phantomsection
\addcontentsline{lot}{section}{\protect\numberline{}Tabel 45 - Cakupan pemberian vitamin A pada bayi dan anak balita}
\ra{1.3}
%Table generated by Excel2LaTeX from sheet '45'

{\centering
\begin{tabular}{rY{9em}Y{8em}Z{4em}Z{4em}Z{4em}Z{4em}Z{4em}Z{4em}Z{4em}Z{4em}Z{4em}}
    \multicolumn{12}{l}{Tabel 45}\\
    \multicolumn{12}{c}{CAKUPAN PEMBERIAN VITAMIN A PADA BAYI DAN ANAK BALITA MENURUT KECAMATAN DAN PUSKESMAS}\\
    \multicolumn{12}{c}{KABUPATEN BELITUNG TIMUR}\\
    \multicolumn{12}{c}{TAHUN \tP}\\
    \toprule
    \multirow{3}[0]{*}{NO} & \multirow{3}[0]{*}{KECAMATAN} & \multirow{3}[0]{*}{PUSKESMAS} & \multicolumn{3}{c}{BAYI 6-11 BULAN\textsuperscript{1}} & \multicolumn{3}{c}{ANAK BALITA (12-59 BULAN)} & \multicolumn{3}{c}{BALITA (6-59 BULAN)} \\
    \cmidrule(l{2pt}r{2pt}){4-6} \cmidrule(l{2pt}r{2pt}){7-9} \cmidrule(l{2pt}r{2pt}){10-12}
    & & & \multirow{3}[0]{4em}{\raggedleft JUMLAH BAYI} & \multicolumn{2}{c}{MENDAPAT VIT A} & \multirow{3}[0]{4em}{JUMLAH } & \multicolumn{2}{c}{MENDAPAT VIT A} & \multirow{3}[0]{4em}{JUMLAH} & \multicolumn{2}{c}{MENDAPAT VIT A} \\
    \cmidrule{5-6}\cmidrule{8-9}\cmidrule{11-12}
%    & & & & \multicolumn{1}{c}{$\Sigma$} & \multicolumn{1}{c}{\%} & & \multicolumn{1}{c}{$\Sigma$} & \multicolumn{1}{c}{\%} & & \multicolumn{1}{c}{$\Sigma$} & \multicolumn{1}{c}{\%} \\
    & & & & Jml & \% & & Jml & \% & & Jml & \% \\
    \midrule
    \emph{1} & \emph{2} & \emph{3} & \emph{4} & \emph{5} & \emph{6} & \emph{7} & \emph{8} & \emph{9} & \emph{10} & \emph{11} & \emph{12} \\
    \midrule
	1 & Manggar           & Manggar       &   617 &   515 &  83,47 & 2.345 & 1.930 &  82,30 & 2.962 & 2.445 & 82,55 \\
	2 & Damar             & Mengkubang    &   207 &   155 &  74,88 &   740 &   661 &  89,32 &   947 &   816 & 86,17 \\
	3 & Kelapa Kampit     & Kelapa Kampit &   295 &   266 &  90,17 &   964 &   936 &  97,10 & 1.259 & 1.202 & 95,47 \\
	4 & Gantung           & Gantung       &   454 &   488 & 107,49 & 1.833 & 1.481 &  80,80 & 2.287 & 1.969 & 86,10 \\
	5 & Simpang Renggiang & Renggiang     &   118 &    90 &  76,27 &   422 &   438 & 103,79 &   540 &   528 & 97,78 \\
	6 & Simpang Pesak     & Simpang Pesak &   133 &   111 &  83,46 &   506 &   471 &  93,08 &   639 &   582 & 91,08 \\
	7 & Dendang           & Dendang       &   167 &   128 &  76,65 &   558 &   435 &  77,96 &   725 &   563 & 77,66 \\
    \midrule
    \multicolumn{3}{l}{JUMLAH}            & 1.991 & 1.753 &  88,05 & 7.368 & 6.352 &  86,21 & 9.359 & 8.105 & 86,60 \\
    \bottomrule
\end{tabular}%

}

\vspace{2ex}
{\small
	\textsuperscript{1} Keterangan: Pelaporan pemberian vitamin A dilakukan pada Februari dan Agustus, maka perhitungan bayi 6-11 bulan yang mendapat vitamin A dalam setahun dihitung dengan mengakumulasi bayi 6-11 bulan yang mendapat vitamin A di bulan Februari dan yang mendapat vitamin A di bulan Agustus. Sehingga jumlah sasaran bayi 6-11 bulan = jumlah bayi setahun.
	
}
\vfill
Sumber: Subkoordinator Kesehatan Keluarga dan Gizi\par 

