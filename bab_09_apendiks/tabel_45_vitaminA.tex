\phantomsection
\addcontentsline{lot}{section}{\protect\numberline{}Tabel 45 - Cakupan pemberian vitamin A pada bayi dan anak balita}
\ra{1.3}
%Table generated by Excel2LaTeX from sheet '45'

{\centering
\begin{tabular}{rY{9em}Y{8em}Z{4em}Z{4em}Z{4em}Z{4em}Z{4em}Z{4em}Z{4em}Z{4em}Z{4em}}
    \multicolumn{12}{l}{Tabel 45}\\
    \multicolumn{12}{c}{CAKUPAN PEMBERIAN VITAMIN A PADA BAYI DAN ANAK BALITA MENURUT KECAMATAN DAN PUSKESMAS}\\
    \multicolumn{12}{c}{KABUPATEN BELITUNG TIMUR}\\
    \multicolumn{12}{c}{TAHUN \tP}\\
    \toprule
    \multirow{3}[0]{*}{NO} & \multirow{3}[0]{*}{KECAMATAN} & \multirow{3}[0]{*}{PUSKESMAS} & \multicolumn{3}{c}{BAYI 6-11 BULAN\textsuperscript{1}} & \multicolumn{3}{c}{ANAK BALITA (12-59 BULAN)} & \multicolumn{3}{c}{BALITA (6-59 BULAN)} \\
    \cmidrule(l{2pt}r{2pt}){4-6} \cmidrule(l{2pt}r{2pt}){7-9} \cmidrule(l{2pt}r{2pt}){10-12}
    & & & \multirow{3}[0]{4em}{\raggedleft JUMLAH BAYI} & \multicolumn{2}{c}{MENDAPAT VIT A} & \multirow{3}[0]{4em}{JUMLAH } & \multicolumn{2}{c}{MENDAPAT VIT A} & \multirow{3}[0]{4em}{JUMLAH} & \multicolumn{2}{c}{MENDAPAT VIT A} \\
    \cmidrule{5-6}\cmidrule{8-9}\cmidrule{11-12}
%    & & & & \multicolumn{1}{c}{$\Sigma$} & \multicolumn{1}{c}{\%} & & \multicolumn{1}{c}{$\Sigma$} & \multicolumn{1}{c}{\%} & & \multicolumn{1}{c}{$\Sigma$} & \multicolumn{1}{c}{\%} \\
    & & & & Jml & \% & & Jml & \% & & Jml & \% \\
    \midrule
    \emph{1} & \emph{2} & \emph{3} & \emph{4} & \emph{5} & \emph{6} & \emph{7} & \emph{8} & \emph{9} & \emph{10} & \emph{11} & \emph{12} \\
    \midrule
	1 & Manggar           & Manggar       &   401 &   390 &  97,26 & 1.807 & 1.687 &  93,36 & 2.208 & 2.077 &  94,07 \\
	2 & Damar             & Mengkubang    &   190 &   190 & 100,00 &   681 &   681 & 100,00 &   871 &   871 & 100,00 \\
	3 & Kelapa Kampit     & Kelapa Kampit &   260 &   260 & 100,00 &   996 &   996 & 100,00 & 1.256 & 1.256 & 100,00 \\
	4 & Gantung           & Gantung       &   420 &   420 & 100,00 & 1.347 & 1.304 &  96,81 & 1.767 & 1.724 &  97,57 \\
	5 & Simpang Renggiang & Renggiang     &   102 &   102 & 100,00 &   404 &   404 & 100,00 &   506 &   506 & 100,00 \\
	6 & Simpang Pesak     & Simpang Pesak &    99 &    94 &  94,95 &   455 &   440 &  96,70 &   554 &   534 &  96,39 \\
	7 & Dendang           & Dendang       &   134 &   134 & 100,00 &   603 &   596 &  98,84 &   737 &   730 &  99,05 \\
    \midrule
    \multicolumn{3}{l}{JUMLAH}            & 1.606 & 1.590 &  99,00 & 6.293 & 6.108 &  97,06 & 7.899 & 7.698 &  97,46 \\
    \bottomrule
\end{tabular}%

}

\vspace{2ex}
{\small
	\textsuperscript{1} Keterangan: Pelaporan pemberian vitamin A dilakukan pada Februari dan Agustus, maka perhitungan bayi 6-11 bulan yang mendapat vitamin A dalam setahun dihitung dengan mengakumulasi bayi 6-11 bulan yang mendapat vitamin A di bulan Februari dan yang mendapat vitamin A di bulan Agustus. Sehingga jumlah sasaran bayi 6-11 bulan = jumlah bayi setahun.
	
}
\vfill
Sumber: Subkoordinator Kesehatan Keluarga dan Gizi\par 
