\chapter{Standar Pelayanan Minimal}
\begin{center}
\renewcommand*{\arraystretch}{1.3}
%\newcolumntype{R}[1]{>{\raggedleft\arraybackslash}p{#1}}
\begin{longtable}{rY{20em}Z{6em}Z{6em}Z{6em}Z{8em}Z{7em}}
% prevent longtable show up twice in TOC. set caption for firsthead, then set empty starred
% caption for next heads. in longtable \toprule should be prepended by \\
\caption{Capaian Standar Pelayanan Minimal (SPM) Bidang Kesehatan Kab. Belitung Timur Tahun \tP}
\\ \toprule
\multirow{2}[0]{*}{No} & \multirow{2}[0]{*}{JENIS PELAYANAN{*}} & \multicolumn{3}{c}{MUTU LAYANAN} & \multirow{2}[0]{7em}{\raggedleft MUTU BARANG/ JASA/ SDM (\%)} & \multirow{2}[0]{6em}{\raggedleft MUTU SPM (\%){***}}\\
\cmidrule(l{2pt}r{2pt}){3-5}
   &                    & PEMBILANG & PENYEBUT{**} & MUTU (\%) &                             &                   \\
\midrule
\endfirsthead
\caption*{}
\\ \toprule
\multirow{2}[0]{*}{No} & \multirow{2}[0]{*}{JENIS PELAYANAN{*}} & \multicolumn{3}{c}{MUTU LAYANAN} & \multirow{2}[0]{7em}{\raggedleft MUTU BARANG/ JASA/ SDM (\%)} & \multirow{2}[0]{6em}{\raggedleft MUTU SPM (\%){***}}\\
\cmidrule(l{2pt}r{2pt}){3-5}
&                    & PEMBILANG & PENYEBUT{**} & MUTU (\%) &                             &                   \\
\midrule
\endhead

 1 & Pelayanan kesehatan ibu hamil                          &  1.485 &  2.102 &  70,65 &  97,02 & 75,92 \\
 2 & Pelayanan kesehatan ibu bersalin                       &  1.680 &  2.102 &  79,92 & 100,00 & 83,94 \\
 3 & Pelayanan kesehatan bayi baru lahir                    &  1.642 &  2.138 &  76,80 & 100,00 & 81,44 \\
 4 & Pelayanan kesehatan balita                             &  7.742 &  8.409 &  92,07 &  80,95 & 89,84 \\
 5 & Pelayanan kesehatan pada usia pendidikan dasar         & 19.118 & 19.825 &  96,43 &  72,22 & 91,59 \\
 6 & Pelayanan kesehatan pada usia produktif                & 83.086 & 86.746 &  95,78 &  63,16 & 89,26 \\
 7 & Pelayanan kesehatan pada usia lanjut                   & 11.620 & 14.277 &  81,39 & 100,00 & 85,11 \\
 8 & Pelayanan kesehatan penderita hipertensi               & 23.035 & 24.264 &  94,93 &  83,33 & 92,61 \\
 9 & Pelayanan kesehatan penderita diabetes melitus         &  2.151 &  2.419 &  88,92 &  91,67 & 89,47 \\
10 & Pelayanan kesehatan orang dengan gangguan jiwa berat   &    328 &    328 & 100,00 &  66,67 & 93,33 \\
11 & Pelayanan kesehatan orang terduga tuberkulosis         &  2.322 &  2.262 & 102,65 &  80,00 & 96,00 \\
12 & Pelayanan kesehatan orang dengan risiko terinfeksi HIV &  2.518 &  3.101 &  81,20 &  90,00 & 82,96 \\
\midrule
\multicolumn{2}{r}{Indeks SPM{****}}                        &        &        &        &        & 87,62 \\
                                                     \multicolumn{7}{r}{TUNTAS MADYA} \\
\bottomrule
\end{longtable}
\par\end{center}


{*}) \emph{Sesuai Peraturan Menteri Kesehatan Nomor 6 Tahun 2024 tentang Standar Teknis Pemenuhan Mutu Pelayanan Dasar Pada Standar Pelayanan
Minimal Bidang Kesehatan}

{**}) \emph{Berdasarkan estimasi dan tidak selalu menggambarkan jumlah yang sebenarnya di populasi}

{***}) \emph{Dihitung dari komponen mutu layanan dan komponen mutu barang/jasa/SDM}

{****}) \emph{Sesuai Peraturan Menteri Dalam Negeri Nomor 59 Tahun 2021 Tentang Penerapan Standar Pelayanan Minimal}
