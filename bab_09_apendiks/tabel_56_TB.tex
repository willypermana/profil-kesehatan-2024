\phantomsection
\addcontentsline{lot}{section}{\protect\numberline{}Tabel 56 - Jumlah terduga tuberkulosis ,kasus tuberkulosis, kasus tuberkulosis anak dan CNR}
\ra{1.15}

%Table generated by Excel2LaTeX from sheet '56'
{\centering
\begin{small}
\begin{tabular}{rY{9em}Y{8em}rZ{4em}Z{4em}Z{4em}Z{4em}Z{4em}r}
    \multicolumn{10}{l}{Tabel 56}\\
    \multicolumn{10}{c}{JUMLAH TERDUGA TUBERKULOSIS ,KASUS TUBERKULOSIS, KASUS TUBERKULOSIS ANAK,}\\
    \multicolumn{10}{c}{DAN \emph{TREATMENT COVERAGE} (TC) MENURUT JENIS KELAMIN, KECAMATAN, DAN PUSKESMAS}\\
    \multicolumn{10}{c}{KABUPATEN BELITUNG TIMUR}\\
    \multicolumn{10}{c}{TAHUN \tP}\\
    \toprule
    \multirow{3}[0]{*}{NO} & \multirow{3}[0]{*}{KECAMATAN} & \multirow{3}[0]{*}{PUSKESMAS} & \multirow[t]{3}{11em}{\raggedleft JUMLAH TERDUGA TUBERKULOSIS YANG MENDAPATKAN PELAYANAN SESUAI STANDAR} & \multicolumn{5}{c}{JUMLAH SEMUA KASUS TUBERKULOSIS} & \multirow{3}{7em}{\raggedleft KASUS TUBERKULOSIS ANAK 0-14 TAHUN}\\
    %\multirow[vpos]{nrows}[bigstruts]{width}[vmove]{text}
    &&&&&&&&&\\
    \cmidrule{5-9}
    & & & & \multicolumn{2}{c}{L} & \multicolumn{2}{c}{P} & \multirow{2}[0]{*}{L+P} & \\
    \cmidrule(l{2pt}r{2pt}){5-6}\cmidrule(l{2pt}r{2pt}){7-8}
    & & & & Jml & \% & Jml & \% & & \\
    \midrule
    \emph{1} & \emph{2} & \emph{3} & \emph{4} & \emph{5} & \emph{6} & \emph{7} & \emph{8} & \emph{9} & \emph{10} \\
    \midrule
    1 & Manggar           & Manggar                 &   553 &  63 & 68,48 &     29 & 31,52 &    92 &    11 \\
    2 & Damar             & Mengkubang              &   330 &  13 & 48,15 &     14 & 51,85 &    27 &     5 \\
    3 & Kelapa Kampit     & Kelapa Kampit           &   249 &  21 & 53,85 &     18 & 46,15 &    39 &    13 \\
    4 & Gantung           & Gantung                 &   574 &  50 & 66,67 &     25 & 33,33 &    75 &     9 \\
    5 & Simpang Renggiang & Renggiang               &   176 &  10 & 52,63 &      9 & 47,37 &    19 &     3 \\
    6 & Simpang Pesak     & Simpang Pesak           &   191 &  15 & 93,75 &      1 &  6,25 &    16 &     1 \\
    7 & Dendang           & Dendang                 &   259 &   6 & 75,00 &      2 & 25,00 &     8 &     4 \\
    \midrule                                        
    \multicolumn{3}{l}{JUMLAH KAB.}                 & 2.332 & 178 & 64,49 &     98 & 35,51 &   276 &    46 \\
    \midrule
    \multicolumn{3}{l}{JUMLAH TERDUGA TUBERKULOSIS} & 2.039 & & & & & & \\
    \multicolumn{7}{Y{10cm}}{PERSENTASE ORANG TERDUGA TUBERKULOSIS MENDAPATKAN PELAYANAN TUBERKULOSIS SESUAI STANDAR} & 114,37 & & \\
    \midrule
    \multicolumn{8}{l}{PERKIRAAN INSIDEN TUBERKULOSIS (DALAM ABSOLUT) } & 419 & \\
    \multicolumn{8}{l}{\textit{TREATMENT COVERAGE (\%)}} & 65,87 & \\
    \midrule
    \multicolumn{9}{l}{CAKUPAN PENEMUAN KASUS TUBERKULOSIS ANAK (\%)} & 91,49 \\
    \bottomrule
\end{tabular}%
\end{small}

}

\vfill
Sumber: Subkoordinator Pengendalian Penyakit Menular\par
