\chapter{SUMBER DAYA MANUSIA KESEHATAN}
Pelayanan kesehatan kepada masyarakat harus didukung dengan tenaga kesehatan, yang berkompetensi. Untuk menjalankan fungsi pengembangan, Dinas Kesehatan Kabupaten Belitung Timur sebagai fasilitator dan koordinator dalam pendidikan dan pelatihan sumber daya kesehatan dengan kebijakan bahwa semua bentuk pelatihan yang dilaksanakan oleh Dinas Kesehatan Kabupaten Belitung Timur dalam meningkatkan kinerja tenaga kesehatan. Sedangkan di setiap UPTD Puskesmas dan Subbagian/ Bidang berkoordinasi dalam perencanaan dan diklat. Hal ini untuk meningkatkan efesiensi dan efektifitas diklat dan menghindari \textit{overlapping} jenis dan kuantitas diklat.

Pelaksanaan program sumber daya manusia kesehatan bertujuan untuk meningkatkan jumlah, jenis, mutu dan penyebaran tenaga kesehatan serta pemberdayaan profesi kesehatan, yang sesuai dengan kebutuhan. Peningkatan keterampilan dan profesionalisme tenaga kesehatan yaitu dengan pendidikan dan pelatihan tenaga kesehatan, dan menyusun standar kompetensi dan regulasi profesi.

Kebutuhan tenaga kesehatan ditentukan oleh pemenuhan rasio tenaga kesehatan berdasarkan jumlah penduduk pada tingkat kabupaten serta pemenuhan standar ketenagaan minimal pada tingkat fasilitas pelayanan kesehatan (Puskesmas dan Rumah Sakit). Standar rasio tenaga kesehatan berdasarkan jumlah penduduk diatur dalam Keputusan Menteri Koordinator Bidang Kesejahteraan Rakyat Republik Indonesia Nomor 54 Tahun 2013 Tentang Rencana Pengembangan Tenaga Kesehatan Tahun 2011 – 2025. Sedangkan standar ketenagaan minimal pada tingkat fasilitas pelayanan kesehatan
diatur dalam Peraturan Menteri Kesehatan Republik Indonesia Nomor 75 Tahun 2014 Tentang Puskesmas serta Peraturan Menteri Kesehatan Republik Indonesia Nomor 56 Tahun 2014 Tentang Klasifikasi dan Perizinan Rumah Sakit.

Dalam memenuhi SDM kesehatan yang belum memenuhi standar rasio kesehatan penduduk dilakukan pengadaan, penetapan dan penyebaran tenaga kesehatan. Penambahan dan penetapan SDM kesehatan dilakukan kerjasama dengan pihak-pihak yang terkait, antara lain Departemen Kesehatan RI, Pemerintah Kabupaten, Pemerintah Provinsi. Program beasiswa dilakukan terus menerus dalam upaya peningkatan SDM Kesehatan ini. Sumber pembiayaan dari APBN, APBD Tk.I, maupun APBD Tk. II, setiap tahunnya ditargetkan untuk tugas belajar (Tubel) dengan pembagian yang merata di setiap Pusat Kesehatan yang ada di setiap kecamatan.

\section{TENAGA MEDIS}
Undang-undang Nomor 17 Tahun 2023 tentang Kesehatan mengatur bahwa yang termasuk dalam kelompok tenaga medis adalah dokter, dokter gigi, dokter spesialis, dan dokter gigi spesialis.
Dokter dan dokter gigi adalah dokter, dokter spesialis, dokter gigi, dan dokter gigi spesialis lulusan pendidikan kedokteran atau kedokteran gigi baik di dalam maupun di luar negeri yang diakui oleh Pemerintah Republik Indonesia sesuai dengan peraturan perundang-undangan.

Jumlah Dokter Umum di Kabupaten Belitung Timur pada tahun \tP adalah sebanyak 50 (Lima Puluh) orang dengan rasio 38,75 per 100.000 penduduk.
Dokter Spesialis di Kabupaten Belitung Timur tahun \tP berjumlah 17 (Tujuh Belas) orang dengan rasio 13,17 per 100.000 penduduk.
Dokter Gigi (termasuk Dokter Spesialis Gigi) berjumlah 9 (Sembilan) orang dengan rasio 6,97 per 100.000 penduduk.

\section{TENAGA KESEHATAN}
\subsection{Tenaga Keperawatan dan Kebidanan}
Perawat adalah seseorang yang telah lulus pendidikan tinggi keperawatan, baik di dalam maupun di luar negeri yang diakui oleh Pemerintah sesuai dengan ketentuan peraturan perundang-undangan.
Bidan adalah seorang perempuan yang lulus dari pendidikan bidan yang telah teregistrasi sesuai ketentuan peraturan perundang-undangan .

Jumlah tenaga kesehatan Perawat di Kabupaten Belitung Timur tahun \tP sebanyak 361 (Tiga Ratus Enam Puluh Satu) orang dengan rasio 279,74 per 100.000 penduduk.

Jumlah tenaga kesehatan Bidan di Kabupaten Belitung Timur tahun \tP adalah sebanyak 157 (Seratus Lima Puluh Tujuh) orang dengan rasio 121,66 per 100.000 penduduk.

\begin{table}[H]
\caption{Rasio Tenaga Kesehatan di Kab. Belitung Timur tahun \tP}
\label{tab:Rasio-Tenaga-Kesehatan}
\centering{}%
\ra{1.3}
\renewcommand*{\arraystretch}{1.3}

\begin{tabular}{rY{0.3\textwidth}rrr}
\toprule
\multicolumn{1}{c}{No} & \makecell[l]{Jenis\\Tenaga Kesehatan} & Jumlah & \makecell[r]{Rasio Tahun\\\tP (per 100.000\\penduduk)}
& \makecell[r]{Target Rasio Tahun\\\tP\footnotemark[1] (per 100.000\\ penduduk)}\\
\midrule
    1                     & Dokter Spesialis                         &  18 &  13,17 &  11 \\
    \rowcolor{black!10}2  & Dokter Umum                              &  62 &  38,75 &  45 \\
                       3  & Dokter Gigi                              &   9 &   6,97 &  13 \\
    \rowcolor{black!10}4  & Perawat                                  & 370 & 279,74 & 180 \\
                       5  & Bidan                                    & 179 & 121,66 & 120 \\
    \rowcolor{black!10}6  & Apoteker                                 &  15 &  11,62 &  12 \\
                       7  & Tenaga Teknis Kefarmasian                &  26 &  16,27 &  24 \\
    \rowcolor{black!10}8  & Tenaga Kesehatan Masyarakat              &  19 &  14,72 &  15 \\
                       9  & Tenaga Kesehatan Lingkungan              &  13 &   7,75 &  18 \\
    \rowcolor{black!10}10 & Tenaga Gizi                              &  22 &  17,82 &  14 \\
                       11 & Tenaga Ahli Teknologi Laboratorium Medik &  27 &  20,92 & N/A \\
    \rowcolor{black!10}12 & Tenaga Teknik Biomedika Lainnya          &  10 &   8,52 & N/A \\
                       13 & Tenaga Keterapian Fisik                  &   7 &   5,42 &   5 \\
    \rowcolor{black!10}14 & Tenaga Keteknisian Medis                 &  34 &  26,43 &  16\\
\bottomrule
\end{tabular}
\end{table}
\footnotetext[1]{Target Nasional RPTK Tahun 2011-2025 (Kepmenko Kesra No.54 Tahun 2013)}

\subsection{Tenaga Kesehatan Masyarakat, Kesehatan Lingkungan dan Tenaga Gizi}
Tenaga kesehatan masyarakat adalah tenaga kesehatan yang telah memenuhi kualifikasi bidang kesehatan masyarakat
yang terdiri dari epidemiolog kesehatan, tenaga promosi kesehatan dan ilmu perilaku, pembimbing kesehatan kerja,
tenaga administrasi dan kebijakan kesehatan, tenaga biostatistik dan kependudukan, serta tenaga kesehatan reproduksi
dan keluarga sesuai dengan peraturan perundang-undangan yang berlaku.
Tenaga kesehatan lingkungan adalah tenaga kesehatan yang telah memenuhi kualifikasi bidang kesehatan lingkungan yang terdiri dari sanitasi lingkungan,
entomolog kesehatan, mikrobiolog kesehatan sesuai dengan peraturan perundang-undangan yang berlaku.
Tenaga gizi adalah tenaga kesehatan yang telah memenuhi kualifikasi bidang gizi yang terdiri dari nutrisionis dan dietisien
sesuai dengan peraturan perundang-undangan yang berlaku.

Jumlah tenaga Kesehatan Masyarakat berjumlah 19 (Sembilan Belas) orang dengan rasio 14,72 per 100.000 penduduk, tenaga Kesehatan Lingkungan sebanyak 10 (Sepuluh) orang dengan rasio 7,75 per 100.000 penduduk, dan tenaga Gizi berjumlah 23 (Dua Puluh Tiga) orang dengan rasio 17,82 per 100.000 penduduk.

\subsection{Tenaga Teknik Biomedika, Keterapian Fisik dan Keteknisan Medik}
Tenaga ahli teknologi laboratorium medik adalah setiap orang yang telah lulus pendidikan teknologi laboratorium medik atau analis
kesehatan atau analis medis dan memiliki kompetensi melakukan analisis terhadap cairan dan jaringan tubuh manusia untuk
menghasilkan informasi tentang kesehatan perseorangan dan masyarakat sesuai dengan ketentuan peraturan perundang-undangan.
Tenaga teknik biomedika lainnya adalah tenaga kesehatan yang telah memenuhi kualifikasi bidang teknik biomedika yang terdiri dari radiografer, elektromedis, fisikawan medik, radioterapis, dan ortotik prostetik.

Tenaga keterapian fisik adalah tenaga kesehatan yang telah memenuhi kualifikasi bidang keterapian fisik yang terdiri
dari fisioterapis, okupasi terapis, terapis wicara, dan akupunktur sesuai dengan peraturan perundang-undangan yang
berlaku.
Tenaga keteknisian medis adalah tenaga kesehatan yang telah memenuhi kualifikasi bidang keteknisian medis yang
terdiri dari perekam medis dan informasi kesehatan, teknik kardiovaskuler, teknisi pelayanan darah, refraksionis
optisien/ optometris, teknisi gigi, penata anestesi (perawat anastesi), terapis gigi dan mulut (perawat gigi), dan audiologis.

Jumlah tenaga kesehatan Ahli Teknologi Laboratorium Medik di Kabupaten Belitung Timur tahun
\tP adalah sebanyak 27 (Dua Puluh Tujuh) orang dengan rasio 20,92 per 100.000 penduduk.
Jumlah tenaga kesehatan Tenaga Teknik Biomedika Lainnya adalah sebanyak 11 (Sepuluh) orang dengan rasio 8,52 per 100.000 penduduk.

Jumlah tenaga Keterapian Fisik adalah sebanyak 7 (Tujuh) orang dengan rasio 5,42 per 100.000 penduduk. Sedangkan jumlah tenaga Keteknisian Medis adalah 37 (Tiga Puluh Tujuh) orang dengan rasio 28,67 per 100.000 penduduk.


\subsection{Tenaga Kefarmasian}
Tenaga kefarmasian adalah tenaga kesehatan yang telah memenuhi kualifikasi bidang kefarmasian yang terdiri dari apoteker dan tenaga teknis kefarmasian sesuai dengan peraturan perundang-undangan yang berlaku.
Apoteker adalah Sarjana Farmasi yang telah lulus sebagai Apoteker dan telah mengucapkan sumpah jabatan Apoteker.
Tenaga Teknis Kefarmasian adalah tenaga yang membantu Apoteker dalam menjalankan pekerjaan kefarmasian, yang terdiri atas Sarjana Farmasi, Ahli Madya Farmasi, Analis Farmasi dan Tenaga Menengah Farmasi/ Asisten Apoteker.

Jumlah Apoteker di Kabupaten Belitung Timur di tahun \tP adalah sebanyak 15 (Lima Belas) orang dengan rasio 11,62 per 100.000 penduduk. Sedangkan jumlah tenaga teknis kefarmasian adalah 21 (Dua Puluh Satu) orang dengan rasio 16,27 per 100.000 penduduk.

\vspace{2ex}

Rincian lebih lengkap mengenai jumlah tenaga kesehatan di Kabupaten Belitung Timur pada tahun \tP dapat dilihat pada Lampiran Tabel Profil (tabel 13-17).
